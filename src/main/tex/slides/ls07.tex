%%%%%%%%%%%%%%%%%%%%%%%%%%%%%%%%%%%%%%%%%%%%%%%%%%%%%%%%%%%%%%%%%%%%%%%%%%%%%%
% CS240: Programming in C
% Copyright 2016 Pejman Ghorbanzade <mail@ghorbanzade.com>
% Creative Commons Attribution-ShareAlike 4.0 International License
% https://github.com/ghorbanzade/UMB-CS240-2016S/blob/master/LICENSE
%%%%%%%%%%%%%%%%%%%%%%%%%%%%%%%%%%%%%%%%%%%%%%%%%%%%%%%%%%%%%%%%%%%%%%%%%%%%%%

\def \topDirectory {../..}
\def \texDirectory {\topDirectory/src/main/tex}
\def \resDirectory {\topDirectory/src/main/c/res}

\documentclass[compress]{beamer}
%\mode<presentation>
%\usetheme{default}

\usepackage{\texDirectory/template/style/directives}
%%%%%%%%%%%%%%%%%%%%%%%%%%%%%%%%%%%%%%%%%%%%%%%%%%%%%%%%%%%%%%%%%%%%%%%%%%%%%%
% CS240: Programming in C
% Copyright 2015 Pejman Ghorbanzade <pejman@ghorbanzade.com>
% Creative Commons Attribution-ShareAlike 4.0 International License
% https://github.com/ghorbanzade/UMB-CS114-2015F/blob/master/LICENSE
%%%%%%%%%%%%%%%%%%%%%%%%%%%%%%%%%%%%%%%%%%%%%%%%%%%%%%%%%%%%%%%%%%%%%%%%%%%%%%

\course{id}{CS240}
\course{name}{Programming in C}
\course{venue}{Mon/Wed, 5:30 PM - 6:45 PM}
\course{semester}{Spring 2016}
\course{department}{Department of Computer Science}
\course{university}{University of Massachusetts Boston}

\instructor{name}{Pejman Ghorbanzade}
\instructor{title}{}
\instructor{position}{Student Instructor}
\instructor{email}{pejman@cs.umb.edu}
\instructor{phone}{617-287-6419}
\instructor{office}{S-3-124B}
\instructor{office-hours}{Mon/Wed 16:00-17:30}
\instructor{address}{University of Massachusetts Boston, 100 Morrissey Blvd., Boston, MA}

\usepackage{\texDirectory/template/style/beamerthemePejman}
\doc{number}{7}
%\setbeamertemplate{footline}[text line]{}

\begin{document}

\prepareCover

\section{Functions}

\begin{slide}
	\begin{block}{Definition}

	A function is a block of statements that can be called for execution and may return a value.

	\begin{terminal}
	@*\textit{data\_type}*@ @*\textit{function\_name}*@(@*\textit{param\_list}*@)
	{
	    @*\textit{statements}*@;
	}
	\end{terminal}

	\end{block}
\end{slide}

\begin{slide}
	\begin{block}{Declaration}

	\inputminted[firstline=15,lastline=18,linenos]{c}{
		\resDirectory/ls07-function.c
	}

	\end{block}
\end{slide}

\begin{slide}
	\begin{block}{Declaration}

	\inputminted[fontsize=\scriptsize,firstline=24,lastline=37,linenos]{c}{
		\resDirectory/ls07-function.c
	}

	\end{block}
\end{slide}

\begin{slide}
	\begin{block}{Calling a Function}

	\inputminted[fontsize=\small,firstline=43,lastline=50,linenos]{c}{
		\resDirectory/ls07-function.c
	}

	\end{block}
\end{slide}

\begin{slide}
	\begin{block}{Notes}

	\begin{itemize}
	\item[] Functions may have any number of statements.
	\item[] Functions may have any number of parameters.
	\item[] Functions may return only one value.
	\item[] Functions cannot be nested.
	\item[] Functions must be introduced to compiler before they are called.
	\end{itemize}

	\end{block}
\end{slide}

\begin{slide}
	\begin{block}{Coding Style Convention}

	\begin{quotation} \scriptsize \normalfont

	Global variables need to have descriptive names, as do global functions.
	If you have a function that counts the number of active users, you should
	call that ``\texttt{count\_active\_users()}'' or similar. You should \textbf{not} call it ``cntusr()''.

	\begin{flushright}-- Linux Kernel Coding Style\end{flushright}
	
	\end{quotation}

	\end{block}
\end{slide}

\begin{slide}
	\begin{block}{Coding Style Convention}

	\begin{quotation} \scriptsize \normalfont

	Functions should be short and sweet, and do just one thing.
	They should fit on one or two screenfuls of text, and do one thing and do that well.

	The maximum length of a function is inversely proportional to the
	complexity and indentation level of that function.
	[...]
	
	However, if you have a complex function, and you suspect that a less-than-gifted first-year high-school student might not even understand what the function is all about, you should adhere to the maximum limits all the more closely.

	\begin{flushright}-- Linux Kernel Coding Style\end{flushright}
	
	\end{quotation}

	\end{block}
\end{slide}

\begin{slide}
	\begin{block}{Coding Style Convention}

	\begin{quotation} \scriptsize \normalfont

	Another measure of the function is the number of local variables.
	They shouldn't exceed 5-10, or you're doing something wrong.
	Re-think the function, and split it into smaller pieces.
	A human brain can generally easily keep track of about 7 different things, anything more and it gets confused.
	You know you're brilliant, but maybe you'd like to understand what you did 2 weeks from now.

	\begin{flushright}-- Linux Kernel Coding Style\end{flushright}
	
	\end{quotation}

	\end{block}
\end{slide}

\section{User Input}

\begin{slide}
	\begin{block}{Question 1}

	Write a program \texttt{getchar.c} that prompts for name of the user and prints a greeting message using his name.
	Assume user's name is less than 32 characters.

	\end{block}
\end{slide}

\begin{slide}
	\begin{block}{Solution}

	\inputminted[fontsize=\scriptsize,firstline=10,lastline=27,linenos]{c}{
		\resDirectory/ls07-getchar.c
	}

	\end{block}
\end{slide}

\begin{slide}
	\begin{block}{Question 2}

	Write a program \texttt{getint.c} that prompts for a number $n$ and prints $n^2$ as output.
	Assume user input is always a positive integer with less than 16 characters.

	\end{block}
\end{slide}

\begin{slide}
	\begin{block}{Solution}

	\inputminted[fontsize=\scriptsize,firstline=10,lastline=22,linenos]{c}{
		\resDirectory/ls07-getint.c
	}

	\end{block}
\end{slide}

\begin{slide}
	\begin{block}{Solution}

	\inputminted[fontsize=\scriptsize,firstline=24,lastline=39,linenos]{c}{
		\resDirectory/ls07-getint.c
	}

	\end{block}
\end{slide}

\begin{slide}
	\begin{figure}
	\includegraphics[width=0.7\textwidth]{\texDirectory/img/pulp.jpg}
	\end{figure}
\end{slide}

\end{document}
