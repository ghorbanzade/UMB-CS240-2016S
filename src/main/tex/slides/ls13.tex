%%%%%%%%%%%%%%%%%%%%%%%%%%%%%%%%%%%%%%%%%%%%%%%%%%%%%%%%%%%%%%%%%%%%%%%%%%%%%%
% CS240: Programming in C
% Copyright 2016 Pejman Ghorbanzade <mail@ghorbanzade.com>
% Creative Commons Attribution-ShareAlike 4.0 International License
% https://github.com/ghorbanzade/UMB-CS240-2016S/blob/master/LICENSE
%%%%%%%%%%%%%%%%%%%%%%%%%%%%%%%%%%%%%%%%%%%%%%%%%%%%%%%%%%%%%%%%%%%%%%%%%%%%%%

\def \topDirectory {../..}
\def \texDirectory {\topDirectory/src/main/tex}
\def \resDirectory {\topDirectory/src/main/c/res/ls13}

\documentclass[compress]{beamer}
%\mode<presentation>
%\usetheme{default}

\usepackage{\texDirectory/template/style/directives}
%%%%%%%%%%%%%%%%%%%%%%%%%%%%%%%%%%%%%%%%%%%%%%%%%%%%%%%%%%%%%%%%%%%%%%%%%%%%%%
% CS114: Introduction to Programming in Java
% Copyright 2015 Pejman Ghorbanzade <mail@ghorbanzade.com>
% Creative Commons Attribution-ShareAlike 4.0 International License
% https://github.com/ghorbanzade/UMB-CS114-2015F/blob/master/LICENSE
%%%%%%%%%%%%%%%%%%%%%%%%%%%%%%%%%%%%%%%%%%%%%%%%%%%%%%%%%%%%%%%%%%%%%%%%%%%%%%

\course{id}{CS240}
\course{name}{Programming in C}
\course{venue}{Mon/Wed, 5:30 PM - 6:45 PM}
\course{semester}{Spring 2016}
\course{department}{Department of Computer Science}
\course{university}{University of Massachusetts Boston}

\instructor{name}{Pejman Ghorbanzade}
\instructor{title}{}
\instructor{position}{Student Instructor}
\instructor{email}{pejman@cs.umb.edu}
\instructor{phone}{617-287-6419}
\instructor{office}{S-3-124B}
\instructor{office-hours}{Mon/Wed 16:00-17:30}
\instructor{address}{University of Massachusetts Boston, 100 Morrissey Blvd., Boston, MA}

\usepackage{\texDirectory/template/style/beamerthemePejman}
\doc{number}{13}
%\setbeamertemplate{footline}[text line]{}

\begin{document}

\prepareCover

\section{Build Automation}

\subsection{Introduction}

\begin{slide}
	\begin{block}{Objective}

	Write a program \texttt{hello-world.c} that uses \texttt{hello-world.h} to print ''Hello World!'' as standard output.

	\end{block}
\end{slide}

\begin{slide}
	\begin{block}{Implementation}

	\inputminted[fontsize=\footnotesize, firstline=10, linenos]{c}{
		\resDirectory/hello.c
	}

	\end{block}
\end{slide}

\begin{slide}
	\begin{block}{Implementation}


	\inputminted[fontsize=\footnotesize, firstline=10, linenos]{c}{
		\resDirectory/hello.h
	}

	\end{block}
\end{slide}

\begin{slide}
	\begin{block}{Compilation}

	Instructions to build source code:

	\begin{terminal}
	$ ls
	hello.c
	$ gcc -c -o hello.o hello.o
	$ gcc -o hello hello.o
	$ ls
	hello hello.c hello.o
	\end{terminal}

	\end{block}
\end{slide}

\begin{slide}
	\begin{block}{Compilation}

	Instructions to remove build artifacts:

	\begin{terminal}
	$ ls
	hello hello.c hello.o
	$ rm hello
	$ rm hello.o
	$ ls
	hello.c
	\end{terminal}

	\end{block}
\end{slide}

\begin{slide}
	\begin{block}{Problem Statement}

	As software grows in size and complexity, instructions to build the software from source code grow in size.

	\end{block}
\end{slide}

\begin{slide}
	\begin{block}{Proposed Solution}

	Automate the build procedure such that the entire software is built using a simple instruction.

	\end{block}
\end{slide}

\begin{slide}
	\begin{block}{Advantages}

	\begin{itemize}
	\item[] Effortless generation of executable files from the source code
	\item[] Troublefree romoval of build artifacts
	\item[] Automatic dependency management and platfrom configuration
	\end{itemize}

	\end{block}
\end{slide}

\begin{slide}
	\begin{block}{Build Components}

	A build automation process typically includes the following steps.

	\begin{itemize}
	\item[] Code Compilation
	\item[] Unit Testing
	\item[] Static Code Analysis
	\item[] Code Coverage Analysis
	\item[] Build Deployment
	\item[] Documentation
	\item[] Software Versioning
	\end{itemize}

	\end{block}
\end{slide}

\begin{slide}
	\begin{block}{Build Automation Tools}

	Some of the notable tools associated with automating build processes are as follows.

	\begin{description}
	\item[C] GNU Make
	\item[Java] Ant, Maven, Gradle
	\item[Python] SCons, Paver
	\item[Ruby] Rake
	\end{description}

	This course provides a brief introduction to GNU Make build tool.

	\end{block}
\end{slide}

\subsection{GNU Make}

\begin{slide}
	\begin{block}{Objective}

	Write a Makefile that automates the build procedure as shown below:

	\begin{terminal}
	$ ls
	hello.c
	$ make
	gcc -c -o hello.o hello.o
	gcc -o hello hello.o
	$ ls
	hello hello.c hello.o
	$ make clean
	rm hello
	rm hello.o
	\end{terminal}

	\end{block}
\end{slide}

\begin{slide}
	\begin{block}{Makefile v1}

	\inputminted[fontsize=\footnotesize, linenos]{text}{
		\resDirectory/make1.txt
	}

	\end{block}
\end{slide}

\begin{slide}
	\begin{block}{Syntax}

	\begin{terminal}
	@*\textit{target}*@: @*\textit{dependency}*@
	    @*\textit{instruction(s)}*@
	\end{terminal}

	\end{block}
\end{slide}

\begin{slide}
	\begin{block}{Makefile v2}

	\inputminted[fontsize=\scriptsize, linenos]{text}{
		\resDirectory/make2.txt
	}

	\end{block}
\end{slide}

\begin{slide}
	\begin{block}{Makefile v3}

	\inputminted[fontsize=\scriptsize, linenos]{text}{
		\resDirectory/make3.txt
	}

	\end{block}
\end{slide}

\begin{slide}
	\begin{block}{Makefile v4 Part 1}

	\inputminted[fontsize=\scriptsize, lastline=13, linenos]{text}{
		\resDirectory/make4.txt
	}

	\end{block}
\end{slide}

\begin{slide}
	\begin{block}{Makefile v4 Part 2}

	\inputminted[fontsize=\scriptsize, firstline=15, linenos]{text}{
		\resDirectory/make4.txt
	}

	\end{block}
\end{slide}

\end{document}
