%%%%%%%%%%%%%%%%%%%%%%%%%%%%%%%%%%%%%%%%%%%%%%%%%%%%%%%%%%%%%%%%%%%%%%%%%%%%%%
% CS240: Programming in C
% Copyright 2016 Pejman Ghorbanzade <mail@ghorbanzade.com>
% Creative Commons Attribution-ShareAlike 4.0 International License
% https://github.com/ghorbanzade/UMB-CS240-2016S/blob/master/LICENSE
%%%%%%%%%%%%%%%%%%%%%%%%%%%%%%%%%%%%%%%%%%%%%%%%%%%%%%%%%%%%%%%%%%%%%%%%%%%%%%

\def \topDirectory {../..}
\def \texDirectory {\topDirectory/src/main/tex}
\def \resDirectory {\topDirectory/src/main/c/res/ls08}

\documentclass[compress]{beamer}
%\mode<presentation>
%\usetheme{default}

\usepackage{\texDirectory/template/style/directives}
%%%%%%%%%%%%%%%%%%%%%%%%%%%%%%%%%%%%%%%%%%%%%%%%%%%%%%%%%%%%%%%%%%%%%%%%%%%%%%
% CS114: Introduction to Programming in Java
% Copyright 2015 Pejman Ghorbanzade <mail@ghorbanzade.com>
% Creative Commons Attribution-ShareAlike 4.0 International License
% https://github.com/ghorbanzade/UMB-CS114-2015F/blob/master/LICENSE
%%%%%%%%%%%%%%%%%%%%%%%%%%%%%%%%%%%%%%%%%%%%%%%%%%%%%%%%%%%%%%%%%%%%%%%%%%%%%%

\course{id}{CS240}
\course{name}{Programming in C}
\course{venue}{Mon/Wed, 5:30 PM - 6:45 PM}
\course{semester}{Spring 2016}
\course{department}{Department of Computer Science}
\course{university}{University of Massachusetts Boston}

\instructor{name}{Pejman Ghorbanzade}
\instructor{title}{}
\instructor{position}{Student Instructor}
\instructor{email}{pejman@cs.umb.edu}
\instructor{phone}{617-287-6419}
\instructor{office}{S-3-124B}
\instructor{office-hours}{Mon/Wed 16:00-17:30}
\instructor{address}{University of Massachusetts Boston, 100 Morrissey Blvd., Boston, MA}

\usepackage{\texDirectory/template/style/beamerthemePejman}
\doc{number}{8}
%\setbeamertemplate{footline}[text line]{}

\begin{document}

\prepareCover

\section{Recursion}

\begin{slide}
	\begin{block}{Definition}

	A recursive function is a function whose block statements include a call to itself.

	\end{block}
\end{slide}

\begin{slide}
	\begin{block}{Example}

	\inputminted[
		fontsize=\scriptsize,
		firstline=10,
		linenos
	]{c}{\resDirectory/hellos.c}

	\end{block}
\end{slide}

\begin{slide}
	\begin{block}{Conditions}

	\begin{itemize}
	\item[] The recursive call chain must terminate at some point.
	\item[] A recursive call must be compulationally less complicated.
	\item[] No two recursive calls should overlap.
	\end{itemize}

	\end{block}
\end{slide}

\begin{slide}
	\begin{block}{Caution}

	\begin{itemize}
	\item[] Numerous recursive calls will cause prohibitive memory allocation resulting in a stack overflow.
	\item[] Recursion simplifies implementation yet increases space complexity.
	\end{itemize}

	\end{block}
\end{slide}

\section{Macros}

\begin{slide}
	\begin{block}{Definition}

	A macro is a fragment of code which has been given a name.
	Whenever the name is used, it is replaced by the content of the macro.

	\begin{terminal}
	#define @*\textit{macro\_name}*@ @*\textit{token\_sequence}*@
	\end{terminal}

	\end{block}
\end{slide}

\begin{slide}
	\begin{block}{Types}

	\begin{description}
	\item[Object-like Macros] Macros that resemble data objects
	\item[Function-like Macros] Macros that resemble function calls
	\end{description}

	\end{block}
\end{slide}

\begin{slide}
	\begin{block}{Object-like Macros}

	An object-like macro is a simple identifier to be replaced by a code fragment.
	It is most commonly used to give symbolic names to constants.

	\begin{terminal}
	#define BUFFER_SIZE 256
	\end{terminal}

	\end{block}
\end{slide}

\begin{slide}
	\begin{block}{Function-like Macros}

	A function-like macro is a macro whose identifier follows a list of parameters.
	When a macro is \textit{expanded}, each use of a parameter in its body is replaced by the tokens of the corresponding argument.

	\begin{terminal}
	#define max(x, y) ((x > y) ? x : y)
	\end{terminal}

	\end{block}
\end{slide}

\begin{slide}
	\begin{block}{Coding Style Convention}

	\begin{itemize}
	\item[] To easily distinguish them from other variables, macro names are written in upper case.
	\item[] To avoid common pitfalls, using function-like macros is often discouraged.
	\end{itemize}

	\end{block}
\end{slide}


\section{Registers}

\begin{slide}
	\begin{block}{Definition}

	Computer processors provide a very limited number of temporary data storage units, called registers.
	Retreiving data for processing from registers is much faster than from memory.
	Hence, storing a heavily used variable in a register allows efficient access to and manipulation of its value.

	\end{block}
\end{slide}

\begin{slide}
	\begin{block}{Usage}

	C provides the \alert{\texttt{register}} keyword to \emph{ask} for a variable to be stored in a register.
	The compiler will fulfill your request, if a free register is available and it agrees that the specified variable should be stored in a register.

	\begin{terminal}
	register int i = 1;
	\end{terminal}

	\end{block}
\end{slide}

\begin{slide}
	\begin{block}{Drawback}

	Once petitioned to be stored in a register, you may not access the address of the variable.

	\end{block}
\end{slide}

\begin{slide}
	\begin{block}{Note}

	Modern compilers will automatically store heavily used variables in registers.

	\end{block}
\end{slide}

\end{document}
