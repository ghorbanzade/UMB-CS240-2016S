%%%%%%%%%%%%%%%%%%%%%%%%%%%%%%%%%%%%%%%%%%%%%%%%%%%%%%%%%%%%%%%%%%%%%%%%%%%%%%
% CS240: Programming in C
% Copyright 2016 Pejman Ghorbanzade <mail@ghorbanzade.com>
% Creative Commons Attribution-ShareAlike 4.0 International License
% https://github.com/ghorbanzade/UMB-CS240-2016S/blob/master/LICENSE
%%%%%%%%%%%%%%%%%%%%%%%%%%%%%%%%%%%%%%%%%%%%%%%%%%%%%%%%%%%%%%%%%%%%%%%%%%%%%%

\def \topDirectory {../..}
\def \texDirectory {\topDirectory/src/main/tex}
\def \resDirectory {\topDirectory/src/main/c/res/ls17}

\documentclass[compress]{beamer}
%\mode<presentation>
%\usetheme{default}

\usepackage{\texDirectory/template/style/directives}
%%%%%%%%%%%%%%%%%%%%%%%%%%%%%%%%%%%%%%%%%%%%%%%%%%%%%%%%%%%%%%%%%%%%%%%%%%%%%%
% CS240: Programming in C
% Copyright 2015 Pejman Ghorbanzade <pejman@ghorbanzade.com>
% Creative Commons Attribution-ShareAlike 4.0 International License
% https://github.com/ghorbanzade/UMB-CS114-2015F/blob/master/LICENSE
%%%%%%%%%%%%%%%%%%%%%%%%%%%%%%%%%%%%%%%%%%%%%%%%%%%%%%%%%%%%%%%%%%%%%%%%%%%%%%

\course{id}{CS240}
\course{name}{Programming in C}
\course{venue}{Mon/Wed, 5:30 PM - 6:45 PM}
\course{semester}{Spring 2016}
\course{department}{Department of Computer Science}
\course{university}{University of Massachusetts Boston}

\instructor{name}{Pejman Ghorbanzade}
\instructor{title}{}
\instructor{position}{Student Instructor}
\instructor{email}{pejman@cs.umb.edu}
\instructor{phone}{617-287-6419}
\instructor{office}{S-3-124B}
\instructor{office-hours}{Mon/Wed 16:00-17:30}
\instructor{address}{University of Massachusetts Boston, 100 Morrissey Blvd., Boston, MA}

\usepackage{\texDirectory/template/style/beamerthemePejman}
\doc{number}{17}
%\setbeamertemplate{footline}[text line]{}

\usepackage{booktabs}

\begin{document}

\prepareCover

\section{Typedef}

\begin{slide}
	\begin{block}{Definition}

	C provides \texttt{typedef} to define new \textit{data type names} for already defined data types.

	\begin{terminal}
	typedef float Length;
	\end{terminal}

	Once a new type name is defined, it can be used interchangeably throughout the program.

	\begin{terminal}
	Length ruler = 10.0F;
	\end{terminal}

	\end{block}
\end{slide}

\begin{slide}
	\begin{block}{Typedef for Structures}

	You may define new type names for any known data type.

	\inputminted[
		fontsize=\footnotesize,
		firstline=18,
		lastline=22,
		linenos
	]{c}{\resDirectory/typedef.h}

	\end{block}
\end{slide}

\begin{slide}
	\begin{block}{Using Typedefs}

	\inputminted[
		fontsize=\scriptsize,
		firstline=10,
		linenos
	]{c}{\resDirectory/typedef.c}

	\end{block}
\end{slide}

\begin{slide}
	\begin{block}{Coding Style Convention}

	\footnotesize

It's a \emph{mistake} to use typedef for structures and pointers.
When you see a

	\begin{terminal}
	vps_t a;
	\end{terminal}

in the source, what does it mean?
In contrast, if it says

	\begin{terminal}
	struct virtual_container *a;
	\end{terminal}

you can actually tell what ``a'' is.

Lots of people think that typedefs ``help readability''. Not so.

	\begin{flushright}
	Linux Kernel Coding Style Document
	\end{flushright}

	\end{block}
\end{slide}

\section{Enum}

\begin{slide}
	\begin{block}{Definition}

	Enumerated types provide a symbolic name to represent one state out of a list of states.

	\end{block}
\end{slide}

\begin{slide}
	\begin{block}{Introduction}

	C provides \texttt{enum} to define an enumerated list of keywords that can be used interchangeably with their order in the list.

	\inputminted[
		fontsize=\scriptsize,
		firstline=12,
		lastline=17,
		linenos
	]{c}{\resDirectory/enum1.c}

	\end{block}
\end{slide}

\begin{slide}
	\begin{block}{Using Enums}

	Values of enum items can be assigned to variables of similar enum type.

	\inputminted[
		fontsize=\scriptsize,
		firstline=21,
		lastline=24,
		linenos
	]{c}{\resDirectory/enum1.c}

	\end{block}
\end{slide}

\begin{slide}
	\begin{block}{Value of Enum Items}

	Value of a keyword in an enumerated list is its numeric index in list, starting at 0.

	\inputminted[
		fontsize=\scriptsize,
		firstline=25,
		lastline=28,
		linenos
	]{c}{\resDirectory/enum1.c}

	\end{block}
\end{slide}

\begin{slide}
	\begin{block}{Typedef for Enum}

	We may define a new type for an enum list.

	\inputminted[
		fontsize=\scriptsize,
		firstline=10,
		lastline=17,
		linenos
	]{c}{\resDirectory/enum2.c}

	\end{block}
\end{slide}

\begin{slide}
	\begin{block}{Using typedef enums}

	\inputminted[
		fontsize=\scriptsize,
		firstline=19,
		linenos
	]{c}{\resDirectory/enum2.c}

	\end{block}
\end{slide}

\begin{slide}
	\begin{figure}
	\includegraphics[width=0.7\textwidth]{\texDirectory/img/enum.jpg}
	\end{figure}
\end{slide}

\begin{slide}
	\begin{block}{Motivation}

	Enums are extremely useful in helping increas readability of the program.
	They are crucial in building simple key-value maps.
	Enums help preserve memory and reduce size of structures.

	\end{block}
\end{slide}

\begin{slide}
	\begin{block}{Example: \texttt{student.h}}

	\inputminted[
		fontsize=\scriptsize,
		firstline=13,
		lastline=28,
		linenos
	]{c}{\resDirectory/student-enum.h}

	\end{block}
\end{slide}

\begin{slide}
	\begin{block}{Example \texttt{student.c}}

	\inputminted[
		fontsize=\scriptsize,
		firstline=10,
		linenos
	]{c}{\resDirectory/student-enum.c}

	\end{block}
\end{slide}

\section{Unions}

\begin{slide}
	\begin{block}{Definition}

	A union is a variable that may hold (at different times) objects of different types and sizes, with the compiler keeping track of size and alignment requirements.

	Unions allow storing different data types in the same memory location.

	\end{block}
\end{slide}

\begin{slide}
	\begin{block}{Union}

	A union may have an arbitrary number of members of any size, but can store only one member at a time.

	Size of a union is as large as size of its largest member.

	\end{block}
\end{slide}

\begin{slide}
	\begin{block}{Syntax}

	\begin{terminal}
	union @*\textit{tag}*@ {
		@*\textit{member\_type\_1 member\_name\_1}*@;
		@*\textit{member\_type\_2 member\_name\_2}*@;
	} @*\textit{name}*@;
	\end{terminal}

	\end{block}
\end{slide}

\begin{slide}
	\begin{block}{Storing types}

	Any member type can be assigned to a union variable, yet the programmer must ensure the type retireved is consistent with the type most recently stored.

	\end{block}
\end{slide}

\begin{slide}
	\begin{block}{Example}

	\inputminted[
		fontsize=\scriptsize,
		firstline=10,
		lastline=19,
		linenos
	]{c}{\resDirectory/union1.c}

	\end{block}
\end{slide}

\begin{slide}
	\begin{block}{Example}

	\inputminted[
		fontsize=\scriptsize,
		firstline=21,
		linenos
	]{c}{\resDirectory/union1.c}

	\begin{terminal}
	10
	20.000000
	hello
	\end{terminal}

	\end{block}
\end{slide}

\begin{slide}
	\begin{block}{Example}

	\inputminted[
		fontsize=\scriptsize,
		firstline=21,
		linenos
	]{c}{\resDirectory/union2.c}

	\begin{terminal}
	1819043176
	1143141483620823940762435584.000000
	hello
	\end{terminal}

	\end{block}
\end{slide}

\begin{slide}
	\begin{block}{Union vs. Structure}

	Unions can be thought of as structures in which all members have offset zero from the base and the structure is large enough to hold its largest member.

	\end{block}
\end{slide}

\begin{slide}
	\begin{figure}
	\includegraphics[width=0.7\textwidth]{\texDirectory/img/final.jpg}
	\end{figure}
\end{slide}

\end{document}
