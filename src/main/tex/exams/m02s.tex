%%%%%%%%%%%%%%%%%%%%%%%%%%%%%%%%%%%%%%%%%%%%%%%%%%%%%%%%%%%%%%%%%%%%%%%%%%%%%%
% CS240: Programming in C
% Copyright 2015 Pejman Ghorbanzade <mail@ghorbanzade.com>
% Creative Commons Attribution-ShareAlike 4.0 International License
% https://github.com/ghorbanzade/UMB-CS240-2016S/blob/master/LICENSE
%%%%%%%%%%%%%%%%%%%%%%%%%%%%%%%%%%%%%%%%%%%%%%%%%%%%%%%%%%%%%%%%%%%%%%%%%%%%%%

\def \topDirectory {.}
\def \texDirectory {\topDirectory/src/main/tex}

\documentclass[12pt,letterpaper,twoside]{article}
\usepackage{\texDirectory/template/style/directives}
\usepackage{\texDirectory/template/style/assignment}
%%%%%%%%%%%%%%%%%%%%%%%%%%%%%%%%%%%%%%%%%%%%%%%%%%%%%%%%%%%%%%%%%%%%%%%%%%%%%%
% CS114: Introduction to Programming in Java
% Copyright 2015 Pejman Ghorbanzade <mail@ghorbanzade.com>
% Creative Commons Attribution-ShareAlike 4.0 International License
% https://github.com/ghorbanzade/UMB-CS114-2015F/blob/master/LICENSE
%%%%%%%%%%%%%%%%%%%%%%%%%%%%%%%%%%%%%%%%%%%%%%%%%%%%%%%%%%%%%%%%%%%%%%%%%%%%%%

\course{id}{CS240}
\course{name}{Programming in C}
\course{venue}{Mon/Wed, 5:30 PM - 6:45 PM}
\course{semester}{Spring 2016}
\course{department}{Department of Computer Science}
\course{university}{University of Massachusetts Boston}

\instructor{name}{Pejman Ghorbanzade}
\instructor{title}{}
\instructor{position}{Student Instructor}
\instructor{email}{pejman@cs.umb.edu}
\instructor{phone}{617-287-6419}
\instructor{office}{S-3-124B}
\instructor{office-hours}{Mon/Wed 16:00-17:30}
\instructor{address}{University of Massachusetts Boston, 100 Morrissey Blvd., Boston, MA}


\usepackage{amsmath}

\begin{document}

\doc{title}{Solution to Midterm Exam}
\doc{date-pub}{Mar 9, 2016 at 5:30 PM}
\doc{date-due}{Mar 9, 2016 at 6:45 PM}
\doc{points}{8}

\prepare{header}

\section*{Question 1}

You are asked to test a \texttt{hello-world.c} program on the Unix system of the CS department.
The following commands have already been executed.

\begin{terminal}
you@yourmachine:~$ ssh you@users.cs.umb.edu
you@users.cs.umb.edu password:
you@itserver6:~$ pwd
/home/you
you@itserver6:~$ ls
cs240
you@itserver6:~$ ls cs240
hw01 hw02 hw03 hello-world.c
\end{terminal}

The objective is to put the \texttt{hello-world.c} file in a new directory with full path \texttt{/home/you/cs240/m01}, display its output and compile and execute it.
Finally, you should remove the produced object file and the binary files.

List all unix commands that you need to issue to fulfill this objective.

\subsection*{Solution}

\begin{terminal}
you@itserver6:~$ cd cs240
you@itserver6:~/cs240$ mkdir m01
you@itserver6:~/cs240$ mv hello-world.c m01
you@itserver6:~/cs240$ cd m01
you@itserver6:~/cs240/m01$ cat hello-world.c
you@itserver6:~/cs240/m01$ gcc hello-world.c -o hello-world
you@itserver6:~/cs240/m01$ ./hello-world
Hello World!
you@itserver6:~/cs240/m01$ rm hello-world.o hello-world
\end{terminal}

\section*{Question 2}

Bellatrix Lestrange, a CS240 student, was asked to write a program that takes an arbitrary number of non-negative integers as command line arguments and prints their average.
A sample run of the program was expected to be as given below.

\begin{terminal}
$ gcc command-line.c -o command-line
$ ./command-line
error: missing command line arguments
$ ./command-line 2 5 8
5.00
$ ./command-line 1 2 3 4 5 6
3.50
\end{terminal}

To solve this problem, she wrote the following program.

\lstset{language=c,tabsize=4}
\lstinputlisting[firstline=10]{\topDirectory/src/main/c/m02/command-line-q.c}

\newpage

Her program produces the following output.

\begin{terminal}
$ gcc command-line.c -o command-line
$ ./command-line
-15202348.00
$ ./command-line 2 5 8
-1147083.00
$ ./command-line 1 2 3 4 5 6
-12761818.00
\end{terminal}

Clearly, something is wrong with Bellatrix's code.
This is why she is asking for your help to debug her program and fix all its bugs.
Rewrite the program \texttt{command-line.c} such that its output matches the expected output provided previously.

\subsection*{Solution}

\lstset{language=c,tabsize=4}
\lstinputlisting[firstline=10]{\topDirectory/src/main/c/m02/command-line.c}

\newpage

\section*{Question 3}

Write a function \texttt{int peak\_to\_peak(int array[], int size)} that returns the difference between minimum and maximum of an array of integers.

\subsection*{Solution}

\lstset{language=c,tabsize=4}
\lstinputlisting[firstline=22]{\topDirectory/src/main/c/m02/peak-to-peak.c}

\end{document}
