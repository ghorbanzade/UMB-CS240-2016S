%%%%%%%%%%%%%%%%%%%%%%%%%%%%%%%%%%%%%%%%%%%%%%%%%%%%%%%%%%%%%%%%%%%%%%%%%%%%%%
% CS240: Programming in C
% Copyright 2015 Pejman Ghorbanzade <mail@ghorbanzade.com>
% Creative Commons Attribution-ShareAlike 4.0 International License
% https://github.com/ghorbanzade/UMB-CS240-2016S/blob/master/LICENSE
%%%%%%%%%%%%%%%%%%%%%%%%%%%%%%%%%%%%%%%%%%%%%%%%%%%%%%%%%%%%%%%%%%%%%%%%%%%%%%

\def \topDirectory {.}
\def \texDirectory {\topDirectory/src/main/tex}

\documentclass[12pt,letterpaper,twoside]{article}
\usepackage{\texDirectory/template/style/directives}
\usepackage{\texDirectory/template/style/assignment}
%%%%%%%%%%%%%%%%%%%%%%%%%%%%%%%%%%%%%%%%%%%%%%%%%%%%%%%%%%%%%%%%%%%%%%%%%%%%%%
% CS114: Introduction to Programming in Java
% Copyright 2015 Pejman Ghorbanzade <mail@ghorbanzade.com>
% Creative Commons Attribution-ShareAlike 4.0 International License
% https://github.com/ghorbanzade/UMB-CS114-2015F/blob/master/LICENSE
%%%%%%%%%%%%%%%%%%%%%%%%%%%%%%%%%%%%%%%%%%%%%%%%%%%%%%%%%%%%%%%%%%%%%%%%%%%%%%

\course{id}{CS240}
\course{name}{Programming in C}
\course{venue}{Mon/Wed, 5:30 PM - 6:45 PM}
\course{semester}{Spring 2016}
\course{department}{Department of Computer Science}
\course{university}{University of Massachusetts Boston}

\instructor{name}{Pejman Ghorbanzade}
\instructor{title}{}
\instructor{position}{Student Instructor}
\instructor{email}{pejman@cs.umb.edu}
\instructor{phone}{617-287-6419}
\instructor{office}{S-3-124B}
\instructor{office-hours}{Mon/Wed 16:00-17:30}
\instructor{address}{University of Massachusetts Boston, 100 Morrissey Blvd., Boston, MA}


\usepackage{amsmath}

\begin{document}

\doc{title}{Solution to Practice Midterm Exam}
\doc{date-pub}{Mar 2, 2016 at 5:30 PM}
\doc{date-due}{Mar 2, 2016 at 6:45 PM}
\doc{points}{8}

\prepare{header}

\section*{Question 1}

You are asked to test a \texttt{hello-world.c} program on the Unix system of the CS department.
The following commands have already been executed.

\begin{terminal}
you@yourmachine:~$ ssh you@users.cs.umb.edu
you@users.cs.umb.edu password:
you@itserver6:~$ pwd
/home/you
you@itserver6:~$ ls cs240
hw01 hw02 hw03
you@itserver6:~$ ls cs240/hw01
hello-world.c hello-world.o hello-world
\end{terminal}

The objective is to remove object file and binary file from \texttt{hw01} directory, create a new directory \texttt{m01} under \texttt{cs240} directory, copy the \texttt{hello-world.c} file there and compile and execute it.
Finally, you should remove the produced object file and the binary files.

List all unix commands that you need to issue to fulfill this objective.

\subsection*{Solution}

\begin{terminal}
you@itserver6:~$ rm cs240/hw01/hello-world.o
you@itserver6:~$ rm cs240/hw01/hello-world
you@itserver6:~$ mkdir cs240/m01
you@itserver6:~$ cp cs240/hw01/hello-world.c cs240/m01
you@itserver6:~$ cd cs240/m01
you@itserver6:~/cs240/m01$ gcc hello-world.c -o hello-world
you@itserver6:~/cs240/m01$ ./hello-world
Hello World!
you@itserver6:~/cs240/m01$ rm hello-world.o hello-world
\end{terminal}

\section*{Question 2}

Edgar Linton, a CS240 student, was asked to write a program that promps user for an integer and prints it on the screen.
A sample run of the program was expected to be as given below.

\begin{terminal}
$ gcc prompt-number.c -o prompt-number
$ ./prompt-number
enter number: 125
number: 125
\end{terminal}

To solve this problem, he wrote the following program.

\lstset{language=c,tabsize=4}
\lstinputlisting[firstline=10]{\topDirectory/src/main/c/m01/prompt-number-q.c}

His program produces the following output.

\begin{terminal}
$ gcc prompt-number.c -o prompt-number
$ ./prompt-number
enter number: 125
number: 971397989
\end{terminal}

\newpage

Clearly, something is wrong with Edgar's code.
This is why he is asking for your help to debug his program and fix all its bugs.
Rewrite the program \texttt{prompt-number.c} such that its output matches the expected output provided previously.

\subsection*{Solution}

\lstset{language=c,tabsize=4}
\lstinputlisting[firstline=10]{\topDirectory/src/main/c/m01/prompt-number.c}

\newpage

\section*{Question 3}

Write a function \texttt{int single\_number(int array[], int size)} that accepts a non-empty array of integers \texttt{array} with size \texttt{size} whose elements appear twice expect for a single element.
Your function should return the element that appears only once.
Following are test cases that clarify the objective.

\begin{terminal}
[1, -1, 4, 5, 4, 1, -1] -> 5
[1, 2, 3, 3, 2] -> 1
[1] -> 1
\end{terminal}

\subsection*{Solution}

\lstset{language=c,tabsize=4}
\lstinputlisting[firstline=21]{\topDirectory/src/main/c/m01/single-number.c}

\end{document}
