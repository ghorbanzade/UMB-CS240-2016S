%%%%%%%%%%%%%%%%%%%%%%%%%%%%%%%%%%%%%%%%%%%%%%%%%%%%%%%%%%%%%%%%%%%%%%%%%%%%%%
% CS240: Programming in C
% Copyright 2015 Pejman Ghorbanzade <mail@ghorbanzade.com>
% Creative Commons Attribution-ShareAlike 4.0 International License
% https://github.com/ghorbanzade/UMB-CS240-2016S/blob/master/LICENSE
%%%%%%%%%%%%%%%%%%%%%%%%%%%%%%%%%%%%%%%%%%%%%%%%%%%%%%%%%%%%%%%%%%%%%%%%%%%%%%

\def \topDirectory {.}
\def \texDirectory {\topDirectory/src/main/tex}

\documentclass[12pt,letterpaper,twoside]{article}
\usepackage{\texDirectory/template/style/directives}
\usepackage{\texDirectory/template/style/assignment}
%%%%%%%%%%%%%%%%%%%%%%%%%%%%%%%%%%%%%%%%%%%%%%%%%%%%%%%%%%%%%%%%%%%%%%%%%%%%%%
% CS114: Introduction to Programming in Java
% Copyright 2015 Pejman Ghorbanzade <mail@ghorbanzade.com>
% Creative Commons Attribution-ShareAlike 4.0 International License
% https://github.com/ghorbanzade/UMB-CS114-2015F/blob/master/LICENSE
%%%%%%%%%%%%%%%%%%%%%%%%%%%%%%%%%%%%%%%%%%%%%%%%%%%%%%%%%%%%%%%%%%%%%%%%%%%%%%

\course{id}{CS240}
\course{name}{Programming in C}
\course{venue}{Mon/Wed, 5:30 PM - 6:45 PM}
\course{semester}{Spring 2016}
\course{department}{Department of Computer Science}
\course{university}{University of Massachusetts Boston}

\instructor{name}{Pejman Ghorbanzade}
\instructor{title}{}
\instructor{position}{Student Instructor}
\instructor{email}{pejman@cs.umb.edu}
\instructor{phone}{617-287-6419}
\instructor{office}{S-3-124B}
\instructor{office-hours}{Mon/Wed 16:00-17:30}
\instructor{address}{University of Massachusetts Boston, 100 Morrissey Blvd., Boston, MA}


\usepackage{amsmath}
\usepackage{amssymb}

\begin{document}

\doc{title}{Assignment 5}
\doc{date-pub}{Apr 4, 2016 at 5:30 PM}
\doc{date-due}{Apr 18, 2016 at 5:30 PM}
\doc{points}{8}

\prepare{header}

\section*{Question 1}

Write a program \texttt{array-generate.c} that takes command line arguments $n$, $m_1$ and $m_2$ and gives as output an array of $n$ integers with values between $m_1$ inclusive and $m_2$ exclusive in random order.

Following is an expected sample run of the program:

\begin{terminal}
$ ls
array-generate.c array-generate.h
$ gcc array-sort.c -o array-sort
$ ./array-generate 10 30 40
34 35 32 38 32 30 33 35 37 30
$ ./array-sort 5 0 10
5 2 8 9 8
\end{terminal}

\section*{Question 2}

Write a program \texttt{array-sort.c} that takes an arbitrary number of integers as command line arguments and prints the array in ascending order randomly choosing of the \texttt{bubble-sort}, \texttt{insertion-sort} or \texttt{selection-sort} algorithms.

\begin{terminal}
$ ls
array-sort.c array-sort.h array-utils.c array-utils.h
$ gcc -c -o array-utils.o array-utils.c 
$ gcc -c -o array-sort.o array-sort.c 
$ gcc -o array-sort array-sort.o array-utils.o -Werror -Wall -std=gnu99 -I.
$ ls
$ ./array-sort 34 35 32 38 32 30 33 35 37 30
30 30 32 32 33 34 35 35 37 38
\end{terminal}

\section*{Question 3 (Optional)}

Write a shell script \texttt{random-sorted-array.sh} that generates three positive random integers (less than 20) and passes them to the \texttt{array-generate} program, developed in Question 1.
The shell script should then pipe the ouput of \texttt{array-generate} as standard input to \texttt{array-sort} program, developed in Question 2.

The desired output of the shell script is given below.

\begin{terminal}
$ ls
array-generate array-sort random-sorted-array.sh
$ chmod 755 random-sorted-array.sh
$ ./random-sorted-array.sh
30 30 32 32 33 34 35 35 37 38
$ ./random-sorted-array.sh
7 7 7 8 8 9 10
\end{terminal}

\section*{Question 4}

Building source code for questions 1 and 2 of this assignment requires initiating multiple commands.
To simplify the build procedure and render the process less error-prone, write a \texttt{Makefile} that summarizes the build procedure as follows.

\begin{terminal}
$ make clean
rm -rf ./array-generate ./array-sort
find . -name '*.o' -delete
$ ls
array-generate.c array-generate.h array-sort.c array-sort.h
array-utils.c array-utils.h random-sorted-array.sh
$ make
gcc -c -o array-generate.o array-generate.c
gcc -c -o array-utils.o array-utils.c
gcc -o array-generate array-generate.o array-utils.o -Werror -Wall -std=gnu99 -I.
gcc -c -o array-sort.o array-sort.c
gcc -o array-sort array-sort.o array-utils.o -Werror -Wall -std=gnu99 -I.
$ ./random-sorted-array.sh
5 12 16 17 17 19 20
\end{terminal}

\prepare{footer}

\end{document}
