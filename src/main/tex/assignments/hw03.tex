%%%%%%%%%%%%%%%%%%%%%%%%%%%%%%%%%%%%%%%%%%%%%%%%%%%%%%%%%%%%%%%%%%%%%%%%%%%%%%
% CS240: Programming in C
% Copyright 2015 Pejman Ghorbanzade <mail@ghorbanzade.com>
% Creative Commons Attribution-ShareAlike 4.0 International License
% https://github.com/ghorbanzade/UMB-CS240-2016S/blob/master/LICENSE
%%%%%%%%%%%%%%%%%%%%%%%%%%%%%%%%%%%%%%%%%%%%%%%%%%%%%%%%%%%%%%%%%%%%%%%%%%%%%%

\def \topDirectory {.}
\def \texDirectory {\topDirectory/src/main/tex}

\documentclass[12pt,letterpaper,twoside]{article}
\usepackage{\texDirectory/template/style/directives}
\usepackage{\texDirectory/template/style/assignment}
%%%%%%%%%%%%%%%%%%%%%%%%%%%%%%%%%%%%%%%%%%%%%%%%%%%%%%%%%%%%%%%%%%%%%%%%%%%%%%
% CS240: Programming in C
% Copyright 2015 Pejman Ghorbanzade <pejman@ghorbanzade.com>
% Creative Commons Attribution-ShareAlike 4.0 International License
% https://github.com/ghorbanzade/UMB-CS114-2015F/blob/master/LICENSE
%%%%%%%%%%%%%%%%%%%%%%%%%%%%%%%%%%%%%%%%%%%%%%%%%%%%%%%%%%%%%%%%%%%%%%%%%%%%%%

\course{id}{CS240}
\course{name}{Programming in C}
\course{venue}{Mon/Wed, 5:30 PM - 6:45 PM}
\course{semester}{Spring 2016}
\course{department}{Department of Computer Science}
\course{university}{University of Massachusetts Boston}

\instructor{name}{Pejman Ghorbanzade}
\instructor{title}{}
\instructor{position}{Student Instructor}
\instructor{email}{pejman@cs.umb.edu}
\instructor{phone}{617-287-6419}
\instructor{office}{S-3-124B}
\instructor{office-hours}{Mon/Wed 16:00-17:30}
\instructor{address}{University of Massachusetts Boston, 100 Morrissey Blvd., Boston, MA}


\usepackage{amsmath}

\begin{document}

\doc{title}{Assignment 3}
\doc{date-pub}{Feb 22, 2016 at 5:30 PM}
\doc{date-due}{Mar 7, 2016 at 5:30 PM}
\doc{points}{8}

\prepare{header}

\section*{Question 1}

Write a program \texttt{bmi.c} that takes your weight in pounds and height in inches and calculates your Body Mass Index (BMI) according to Equation \ref{eq1}.
To evaluate your BMI, program should as well indicate under which group you are classified according to Table \ref{tab1} obtained from the Department of Health and Human Services/National Institution of Health.

\begin{equation}
BMI = \frac{weightInPounds \times 703}{heightInInches^2}
\label{eq1}
\end{equation}

\begin{table}[H]\centering
\begin{tabular}{|r|l|}
\hline
Group & BMI index \\
\hline
Underweight & less than 18.5 \\
Normal & between 18.5 and 24.9 \\
Overweight & between 25 and 29.9 \\
Obese & greater than or equal to 30 \\
\hline
\end{tabular}
\caption{BMI classification}\label{tab1}
\end{table}

Following is the expected output of a sample run of your program.

\begin{terminal}
$ gcc bmi.c -o bmi
$ ./bmi
Your height (in): 72
Your weight (lb): 145
Your BMI is 19.66.
You are classified as normal.
\end{terminal}

\newpage

\section*{Question 2}

Write a program \texttt{reverse.c} that takes an integer number as a command-line argument and prints its reverse.

Your program is expected to function as shown below.

\begin{terminal}
$ gcc reverse.c -o reverse
$ ./reverse
error: missing command line argument
$ ./reverse 12345
Reverse: 54321
$ ./reverse -54321
Reverse: -12345
$ ./reverse hello
error: not a number
$ ./reverse 1234 5678 hello
Reverse: 4321
\end{terminal}

\section*{Question 3}

Write a program \texttt{prime.c} that prints a prime number $n_i$ where $i$ is a positive integer and $n_1 = 2$ and prompts the user whether to print the subsequent prime number.
The program continues to print $n_{i+1}$ as long as the user enters \texttt{Y} or \texttt{y}.
The program terminates when the user enters \texttt{N} or \texttt{n}.

Expected output of a sample run of your program is shown below.

\begin{terminal}
$ gcc prime.c -o prime
$ ./prime
2
another prime [y/n]? Y
3
another prime [y/n]? y
5
another prime [y/n]? YyY
7
another prime [y/n]? t
invalid input.
another prime [y/n]? n
\end{terminal}

\newpage

\prepare{footer}

\end{document}
