%%%%%%%%%%%%%%%%%%%%%%%%%%%%%%%%%%%%%%%%%%%%%%%%%%%%%%%%%%%%%%%%%%%%%%%%%%%%%%
% CS240: Programming in C
% Copyright 2015 Pejman Ghorbanzade <mail@ghorbanzade.com>
% Creative Commons Attribution-ShareAlike 4.0 International License
% https://github.com/ghorbanzade/UMB-CS240-2016S/blob/master/LICENSE
%%%%%%%%%%%%%%%%%%%%%%%%%%%%%%%%%%%%%%%%%%%%%%%%%%%%%%%%%%%%%%%%%%%%%%%%%%%%%%

\def \topDirectory {.}
\def \texDirectory {\topDirectory/src/main/tex}
\def \resDirectory {\topDirectory/src/main/c/hw04}

\documentclass[12pt,letterpaper,twoside]{article}
\usepackage{\texDirectory/template/style/directives}
\usepackage{\texDirectory/template/style/assignment}
%%%%%%%%%%%%%%%%%%%%%%%%%%%%%%%%%%%%%%%%%%%%%%%%%%%%%%%%%%%%%%%%%%%%%%%%%%%%%%
% CS114: Introduction to Programming in Java
% Copyright 2015 Pejman Ghorbanzade <mail@ghorbanzade.com>
% Creative Commons Attribution-ShareAlike 4.0 International License
% https://github.com/ghorbanzade/UMB-CS114-2015F/blob/master/LICENSE
%%%%%%%%%%%%%%%%%%%%%%%%%%%%%%%%%%%%%%%%%%%%%%%%%%%%%%%%%%%%%%%%%%%%%%%%%%%%%%

\course{id}{CS240}
\course{name}{Programming in C}
\course{venue}{Mon/Wed, 5:30 PM - 6:45 PM}
\course{semester}{Spring 2016}
\course{department}{Department of Computer Science}
\course{university}{University of Massachusetts Boston}

\instructor{name}{Pejman Ghorbanzade}
\instructor{title}{}
\instructor{position}{Student Instructor}
\instructor{email}{pejman@cs.umb.edu}
\instructor{phone}{617-287-6419}
\instructor{office}{S-3-124B}
\instructor{office-hours}{Mon/Wed 16:00-17:30}
\instructor{address}{University of Massachusetts Boston, 100 Morrissey Blvd., Boston, MA}


\usepackage{amsmath}

\begin{document}

\doc{title}{Solution to Assignment 4}
\doc{date-pub}{Mar 7, 2016 at 5:30 PM}
\doc{date-due}{Apr 4, 2016 at 5:30 PM}
\doc{points}{8}

\prepare{header}

\section*{Question 1}

Write a program \texttt{fruit-basket.c} that initializes an array of string with names of four different fruits.
The \texttt{main()} function of program is given below and you may not modify it.
In addition, you may not include any header file other than \texttt{fruit-basket.h} in \texttt{fruit-basket.c}.

\lstset{language=c,tabsize=4}
\lstinputlisting[firstline=15,lastline=20]{\resDirectory/fruit-basket.c}

Following is the expected output of a sample run of your program.

\begin{terminal}
$ ls
fruit-basket.c fruit-basket.h
$ gcc -c -o fruit-basket.o fruit-basket.c
$ gcc -o fruit-basket fruit-basket.o -Wall -std=gnu99 -I.
$ ./fruit-basket
fruit 1: banana
fruit 2: apple
fruit 3: orange
fruit 4: pineapple
\end{terminal}

\newpage

\subsection*{Solution}

\lstset{language=c,tabsize=4}
\begin{itemize}
\item[] File \texttt{fruit-basket.c}:
\lstinputlisting[firstline=10]{\resDirectory/fruit-basket.c}
\newpage
\item[] File \texttt{fruit-basket.h}:
\lstinputlisting[firstline=10]{\resDirectory/fruit-basket.h}
\end{itemize}

\section*{Question 2}

Write a program \texttt{slow-prime.c} that takes a positive integer number as a command line argument and prints the smallest prime number bigger than the given number.
The \texttt{main()} function of program is given below and you may not modify it.
In addition, you may not include any header file other than \texttt{slow-prime.h} in \texttt{slow-prime.c}.

\lstset{language=c,tabsize=4}
\lstinputlisting[firstline=16,lastline=34]{\resDirectory/slow-prime.c}

Following is the expected output of a sample run of your program.

\begin{terminal}
$ ls
slow-prime.c slow-prime.h
$ gcc -c -o slow-prime.o slow-prime.c
$ gcc -o slow-prime slow-prime.o -Werror -Wall -std=gnu99 -I.
$ ./slow-prime
error: missing command line argument
$ ./slow-prime 15sa
error: 15sa not a number
$ ./slow-prime 19
23
\end{terminal}

\subsection*{Solution}

\lstset{language=c,tabsize=4}
\begin{itemize}
\item[] File \texttt{slow-prime.c}:
\lstinputlisting[firstline=10]{\resDirectory/slow-prime.c}
\newpage
\item[] File \texttt{slow-prime.h}:
\lstinputlisting[firstline=10]{\resDirectory/slow-prime.h}
\end{itemize}

\section*{Question 3}

Building source code for questions 1 and 2 of this assignment requires initiating multiple commands.
To simplify the build procedure and render the process less error-prone, write a \texttt{Makefile} that summarizes the build procedure as follows.

\begin{terminal}
$ make clean
rm -rf ./fruit-basket ./slow-prime
find . -name '*.o' -delete
$ ls
fruit-basket.c fruit-basket.h Makefile slow-prime.c
slow-prime.h
$ make
gcc -c -o fruit-basket.o fruit-basket.c
gcc -o fruit-basket fruit-basket.o -Wall -std=gnu99 -I.
gcc -c -o slow-prime.o slow-prime.c
gcc -o slow-prime slow-prime.o -Werror -Wall -std=gnu99 -I.
$ ls
fruit-basket fruit-basket.c fruit-basket.h fruit-basket.o
Makefile     slow-prime     slow-prime.c   slow-prime.h
slow-prime.o
\end{terminal}

\newpage

\subsection*{Solution}

\lstset{language=[gnu]make,tabsize=4}
\begin{itemize}
\item[] File \texttt{Makefile}:
\lstinputlisting[firstline=1]{\resDirectory/Makefile}
\end{itemize}

\newpage

\end{document}
