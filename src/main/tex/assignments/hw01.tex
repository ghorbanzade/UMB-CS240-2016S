%%%%%%%%%%%%%%%%%%%%%%%%%%%%%%%%%%%%%%%%%%%%%%%%%%%%%%%%%%%%%%%%%%%%%%%%%%%%%%
% CS240: Programming in C
% Copyright 2015 Pejman Ghorbanzade <mail@ghorbanzade.com>
% Creative Commons Attribution-ShareAlike 4.0 International License
% https://github.com/ghorbanzade/UMB-CS240-2016S/blob/master/LICENSE
%%%%%%%%%%%%%%%%%%%%%%%%%%%%%%%%%%%%%%%%%%%%%%%%%%%%%%%%%%%%%%%%%%%%%%%%%%%%%%

\def \topDirectory {.}
\def \texDirectory {\topDirectory/src/main/tex}

\documentclass[12pt,letterpaper,twoside]{article}
\usepackage{\texDirectory/template/style/directives}
\usepackage{\texDirectory/template/style/assignment}
%%%%%%%%%%%%%%%%%%%%%%%%%%%%%%%%%%%%%%%%%%%%%%%%%%%%%%%%%%%%%%%%%%%%%%%%%%%%%%
% CS114: Introduction to Programming in Java
% Copyright 2015 Pejman Ghorbanzade <mail@ghorbanzade.com>
% Creative Commons Attribution-ShareAlike 4.0 International License
% https://github.com/ghorbanzade/UMB-CS114-2015F/blob/master/LICENSE
%%%%%%%%%%%%%%%%%%%%%%%%%%%%%%%%%%%%%%%%%%%%%%%%%%%%%%%%%%%%%%%%%%%%%%%%%%%%%%

\course{id}{CS240}
\course{name}{Programming in C}
\course{venue}{Mon/Wed, 5:30 PM - 6:45 PM}
\course{semester}{Spring 2016}
\course{department}{Department of Computer Science}
\course{university}{University of Massachusetts Boston}

\instructor{name}{Pejman Ghorbanzade}
\instructor{title}{}
\instructor{position}{Student Instructor}
\instructor{email}{pejman@cs.umb.edu}
\instructor{phone}{617-287-6419}
\instructor{office}{S-3-124B}
\instructor{office-hours}{Mon/Wed 16:00-17:30}
\instructor{address}{University of Massachusetts Boston, 100 Morrissey Blvd., Boston, MA}


\usepackage{amsmath}

\begin{document}

\doc{title}{Assignment 1}
\doc{date-pub}{Jan 27, 2016 at 5:30 PM}
\doc{date-due}{Feb 8, 2016 at 5:30 PM}
\doc{points}{8}

\prepare{header}

\section*{Question 1}

Write a program \texttt{hello-world.c} that prints \textit{``Hello World!"} on the console once executed, as shown below.

\begin{terminal}
$ gcc hello-world.c -o hello-world
$ ./hello-world
Hello World!
\end{terminal}

\section*{Question 2}

Write a program \texttt{weather.c} that takes two arbitrary command-line arguments as the name of a city and its weather condition, respecively, and prints a message in the form given in the following examples.

\begin{terminal}
$ gcc weather.c -o weather
$ ./weather Boston Sunny
Boston is sunny today!
$ gcc weather.c -o weather
$ ./weather Cambridge cloudy
Cambridge is cloudy today!
\end{terminal}

\newpage

\section*{Question 3}

Write a program \texttt{fahrenheit.c} that prints Fahrenheit equivalents of Celcius temperatures 60, 70, 80, 90 and 100 in the following format.

\begin{terminal}
 C    F
---------
 60   140
 70   158
 80   176
 90   194
 100  212
\end{terminal}

\prepare{footer}

\end{document}
