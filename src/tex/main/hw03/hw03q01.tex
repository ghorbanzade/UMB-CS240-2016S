%%%%%%%%%%%%%%%%%%%%%%%%%%%%%%%%%%%%%%%%%%%%%%%%%%%%%%%%%%%%%%%%%%%%%%
% UMB-CS240-2016S: Programming in C
% Copyright 2016 Pejman Ghorbanzade <pejman@ghorbanzade.com>
% Creative Commons Attribution-ShareAlike 4.0 International License
% More info: https://github.com/ghorbanzade/UMB-CS240-2016S
%%%%%%%%%%%%%%%%%%%%%%%%%%%%%%%%%%%%%%%%%%%%%%%%%%%%%%%%%%%%%%%%%%%%%%

\section*{Question 1}

Write a program \texttt{bmi.c} that takes your weight in pounds and height in inches and calculates your Body Mass Index (BMI) according to Equation \ref{eq1}.
To evaluate your BMI, program should as well indicate under which group you are classified according to Table \ref{tab1} obtained from the Department of Health and Human Services/National Institution of Health.

\begin{equation}
BMI = \frac{weightInPounds \times 703}{heightInInches^2}
\label{eq1}
\end{equation}

\begin{table}[H]\centering
\begin{tabular}{|r|l|}
\hline
Group & BMI index \\
\hline
Underweight & less than 18.5 \\
Normal & between 18.5 and 24.9 \\
Overweight & between 25 and 29.9 \\
Obese & greater than or equal to 30 \\
\hline
\end{tabular}
\caption{BMI classification}\label{tab1}
\end{table}

Following is the expected output of a sample run of your program.

\begin{terminal}
$ gcc bmi.c -o bmi
$ ./bmi
Your height (in): 72
Your weight (lb): 145
Your BMI is 19.66.
You are classified as normal.
\end{terminal}
