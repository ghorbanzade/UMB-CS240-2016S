%%%%%%%%%%%%%%%%%%%%%%%%%%%%%%%%%%%%%%%%%%%%%%%%%%%%%%%%%%%%%%%%%%%%%%%%%%%%%%
% CS240: Programming in C
% Copyright 2016 Pejman Ghorbanzade <pejman@ghorbanzade.com>
% Creative Commons Attribution-ShareAlike 4.0 International License
% https://github.com/ghorbanzade/UMB-CS240-2016S/blob/master/LICENSE
%%%%%%%%%%%%%%%%%%%%%%%%%%%%%%%%%%%%%%%%%%%%%%%%%%%%%%%%%%%%%%%%%%%%%%%%%%%%%%

\def \topDirectory {.}
\def \resDirectory {\topDirectory/src/c/main/ls18}
\def \texDirectory {\topDirectory/src/tex}
\def \styDirectory {\texDirectory/sty}
\def \cfgDirectory {\texDirectory/cfg}
\def \imgDirectory {\texDirectory/img}

\documentclass[compress]{beamer}
%\mode<presentation>
%\usetheme{default}

\usepackage{\styDirectory/directives}
%%%%%%%%%%%%%%%%%%%%%%%%%%%%%%%%%%%%%%%%%%%%%%%%%%%%%%%%%%%%%%%%%%%%%%%%%%%%%%
% CS240: Programming in C
% Copyright 2015 Pejman Ghorbanzade <pejman@ghorbanzade.com>
% Creative Commons Attribution-ShareAlike 4.0 International License
% https://github.com/ghorbanzade/UMB-CS114-2015F/blob/master/LICENSE
%%%%%%%%%%%%%%%%%%%%%%%%%%%%%%%%%%%%%%%%%%%%%%%%%%%%%%%%%%%%%%%%%%%%%%%%%%%%%%

\course{id}{CS240}
\course{name}{Programming in C}
\course{venue}{Mon/Wed, 5:30 PM - 6:45 PM}
\course{semester}{Spring 2016}
\course{department}{Department of Computer Science}
\course{university}{University of Massachusetts Boston}

\instructor{name}{Pejman Ghorbanzade}
\instructor{title}{}
\instructor{position}{Student Instructor}
\instructor{email}{pejman@cs.umb.edu}
\instructor{phone}{617-287-6419}
\instructor{office}{S-3-124B}
\instructor{office-hours}{Mon/Wed 16:00-17:30}
\instructor{address}{University of Massachusetts Boston, 100 Morrissey Blvd., Boston, MA}

\usepackage{\styDirectory/beamerthemePejman}
\doc{number}{18}
%\setbeamertemplate{footline}[text line]{}

\usepackage{booktabs}

\begin{document}

\prepareCover

\section{Advanced Data Structures}

\subsection{Motivation}

\begin{slide}
	\begin{block}{Introduction}

	C does not provide built-in functions to support high-level data strucutres.
	It is possible to support any dynamic data structures using structures and pointers.

	\end{block}
\end{slide}

\subsection{ArrayList}

\begin{slide}
	\begin{block}{Definition}

	An array list is a data \emph{structure} that points to a memory block on the heap can store an arbitrary number of elements of a single type.

	\inputminted[
		fontsize=\small,
		firstline=20,
		lastline=24,
		linenos
	]{c}{\resDirectory/array-list.h}

	\end{block}
\end{slide}

\begin{slide}
	\begin{block}{Initializing an ArrayList}

	\inputminted[
		fontsize=\footnotesize,
		firstline=50,
		lastline=54,
		linenos
	]{c}{\resDirectory/array-list.c}

	\end{block}
\end{slide}

\begin{slide}
	\begin{block}{Resizing an ArrayList}

	\inputminted[
		fontsize=\footnotesize,
		firstline=35,
		lastline=45,
		linenos
	]{c}{\resDirectory/array-list.c}

	\end{block}
\end{slide}

\begin{slide}
	\begin{block}{Adding an Element to ArrayList}

	\inputminted[
		fontsize=\footnotesize,
		firstline=50,
		lastline=54,
		linenos
	]{c}{\resDirectory/array-list.c}

	\end{block}
\end{slide}

\begin{slide}
	\begin{block}{Inserting an Element to ArrayList}

	\inputminted[
		fontsize=\scriptsize,
		firstline=106,
		lastline=115,
		linenos
	]{c}{\resDirectory/array-list.c}

	\end{block}
\end{slide}

\begin{slide}
	\begin{block}{Removing an Element from ArrayList}

	\inputminted[
		fontsize=\scriptsize,
		firstline=59,
		lastline=71,
		linenos
	]{c}{\resDirectory/array-list.c}

	\end{block}
\end{slide}

\begin{slide}
	\begin{block}{Removing an Element from ArrayList}

	\inputminted[
		fontsize=\scriptsize,
		firstline=77,
		lastline=93,
		linenos
	]{c}{\resDirectory/array-list.c}

	\end{block}
\end{slide}

\begin{slide}
	\begin{block}{Printing Elements of ArrayList}

	\inputminted[
		fontsize=\footnotesize,
		firstline=159,
		lastline=170,
		linenos
	]{c}{\resDirectory/array-list.c}

	\end{block}
\end{slide}

\subsection{LinkedList}

\begin{slide}
	\begin{block}{Introduction}

	A linked list is a data \emph{structure} of loosely linked memory blocks of same size.

	\inputminted[
		fontsize=\small,
		firstline=14,
		lastline=17,
		linenos
	]{c}{\resDirectory/linked-list.h}

	\end{block}
\end{slide}

\begin{slide}
	\begin{block}{Initializing a Linked List}

	\inputminted[
		fontsize=\small,
		firstline=15,
		lastline=22,
		linenos
	]{c}{\resDirectory/linked-list.c}

	\end{block}
\end{slide}

\begin{slide}
	\begin{block}{Inserting into a Linked List}

	\inputminted[
		fontsize=\scriptsize,
		firstline=39,
		lastline=55,
		linenos
	]{c}{\resDirectory/linked-list.c}

	\end{block}
\end{slide}

\begin{slide}
	\begin{block}{Appending to a Linked List}

	\inputminted[
		fontsize=\small,
		firstline=60,
		lastline=69,
		linenos
	]{c}{\resDirectory/linked-list.c}

	\end{block}
\end{slide}

\begin{slide}
	\begin{block}{Prepending to a Linked List}

	\inputminted[
		fontsize=\small,
		firstline=74,
		lastline=77,
		linenos
	]{c}{\resDirectory/linked-list.c}

	\end{block}
\end{slide}

\begin{slide}
	\begin{block}{Removing from a Linked List}

	\inputminted[
		fontsize=\scriptsize,
		firstline=82,
		lastline=99,
		linenos
	]{c}{\resDirectory/linked-list.c}

	\end{block}
\end{slide}

\begin{slide}
	\begin{block}{Deallocating a Linked List}

	\inputminted[
		fontsize=\small,
		firstline=27,
		lastline=34,
		linenos
	]{c}{\resDirectory/linked-list.c}

	\end{block}
\end{slide}

\begin{slide}
	\begin{figure}
	\includegraphics[width=0.6\textwidth]{\imgDirectory/final-exam.jpg}
	\end{figure}
\end{slide}

\end{document}
