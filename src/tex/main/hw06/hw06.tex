%%%%%%%%%%%%%%%%%%%%%%%%%%%%%%%%%%%%%%%%%%%%%%%%%%%%%%%%%%%%%%%%%%%%%%%%%%%%%%
% CS240: Programming in C
% Copyright 2015 Pejman Ghorbanzade <mail@ghorbanzade.com>
% Creative Commons Attribution-ShareAlike 4.0 International License
% https://github.com/ghorbanzade/UMB-CS240-2016S/blob/master/LICENSE
%%%%%%%%%%%%%%%%%%%%%%%%%%%%%%%%%%%%%%%%%%%%%%%%%%%%%%%%%%%%%%%%%%%%%%%%%%%%%%

\def \topDirectory {.}
\def \texDirectory {\topDirectory/src/main/tex}

\documentclass[12pt,letterpaper,twoside]{article}
\usepackage{\texDirectory/template/style/directives}
\usepackage{\texDirectory/template/style/assignment}
%%%%%%%%%%%%%%%%%%%%%%%%%%%%%%%%%%%%%%%%%%%%%%%%%%%%%%%%%%%%%%%%%%%%%%%%%%%%%%
% CS114: Introduction to Programming in Java
% Copyright 2015 Pejman Ghorbanzade <mail@ghorbanzade.com>
% Creative Commons Attribution-ShareAlike 4.0 International License
% https://github.com/ghorbanzade/UMB-CS114-2015F/blob/master/LICENSE
%%%%%%%%%%%%%%%%%%%%%%%%%%%%%%%%%%%%%%%%%%%%%%%%%%%%%%%%%%%%%%%%%%%%%%%%%%%%%%

\course{id}{CS240}
\course{name}{Programming in C}
\course{venue}{Mon/Wed, 5:30 PM - 6:45 PM}
\course{semester}{Spring 2016}
\course{department}{Department of Computer Science}
\course{university}{University of Massachusetts Boston}

\instructor{name}{Pejman Ghorbanzade}
\instructor{title}{}
\instructor{position}{Student Instructor}
\instructor{email}{pejman@cs.umb.edu}
\instructor{phone}{617-287-6419}
\instructor{office}{S-3-124B}
\instructor{office-hours}{Mon/Wed 16:00-17:30}
\instructor{address}{University of Massachusetts Boston, 100 Morrissey Blvd., Boston, MA}


\usepackage{amsmath}
\usepackage{amssymb}

\begin{document}

\doc{title}{Final Project}
\doc{date-pub}{Apr 4, 2016 at 5:30 PM}
\doc{date-due}{May 2, 2016 at 5:30 PM}
\doc{points}{20}

\prepare{header}

\section*{Question 1}

To check whether a given non-negative integer $n$ is a prime, it suffices to check if it is divisible by at least one of the elements in $\mathbb{S}_n =  \{ k_i \in \mathbb{P} | k_i < \sqrt{n} \}$.
If $\mathbb{S}_n$ is known, it takes $|\mathbb{S}_n|$ operations to check for primality of $n$.
This is much better than when $\mathbb{S}_n$ is not known in which case it takes $\mathbb{O}(n \log \log n)$ operations to construct it.

Suppose we can store $\mathbb{P} = \mathbb{S}_n$, as set of all \textit{known} prime numbers, in a file and access it when testing primality of another non-negative integer $m$.
This way, if $1 << m < n$, primality testing of $m$ would be trivial.
On the other hand, if $1 << n < m$, $\mathbb{S}_n$ would still be very useful in constructing $\mathbb{S}_m$ and required operations for prime-checking $m$ will be reduced by $c_1 n \log \log n + c_2$ operations where $c_1$ and $c_2$ are positive integers.
In this case, once $m$ is prime-checked, we can update the list of \textit{known} prime numbers to $\mathbb{S}_m$ for later use.

You are asked to write a program \texttt{fast-prime.c} that accepts a set of non-negative integer numbers as command line arguments and, using the above-mentioned algorithm, checks for their primality.

You may not include any header file other than \texttt{fast-prime.h}.
In addition, your \texttt{fast-prime.c} file should accompany a \texttt{Makefile} to facilitate building process and artifact removal.

Following illustrates the expected functionality of your program.

\newpage

\begin{terminal}
$ ls
fast-prime.c fast-prime.h Makefile
$ make
gcc -c -o fast-prime.o fast-prime.c
gcc -o fast-prime fast-prime.o -Werror -Wall -std=gnu99 -I.
$ ./fast-prime
error: missing command line argument
$ ./fast-prime 6
6     : not prime
$ ls
fast-prime fast-prime.c fast-prime.h fast-prime.o
Makefile primes.log
$ cat primes.log
2
3
5
$ ./fast-prime 25 37 39 29
25    : not prime
37    : prime
39    : not prime
29    : prime
$ ls
fast-prime fast-prime.c fast-prime.h fast-prime.o Makefile primes.log
$ cat primes.log
2
3
5
7
11
13
17
19
23
29
31
37
$ make clean
rm -rf fast-prime
rm -rf primes.log
find . -name '*.o' -delete
\end{terminal}

\prepare{footer}

\end{document}
