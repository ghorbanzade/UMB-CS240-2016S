%%%%%%%%%%%%%%%%%%%%%%%%%%%%%%%%%%%%%%%%%%%%%%%%%%%%%%%%%%%%%%%%%%%%%%%%%%%%%%
% CS240: Programming in C
% Copyright 2016 Pejman Ghorbanzade <mail@ghorbanzade.com>
% Creative Commons Attribution-ShareAlike 4.0 International License
% https://github.com/ghorbanzade/UMB-CS240-2016S/blob/master/LICENSE
%%%%%%%%%%%%%%%%%%%%%%%%%%%%%%%%%%%%%%%%%%%%%%%%%%%%%%%%%%%%%%%%%%%%%%%%%%%%%%

\def \topDirectory {../..}
\def \texDirectory {\topDirectory/src/main/tex}
\def \resDirectory {\topDirectory/src/main/c/ls16}

\documentclass[compress]{beamer}
%\mode<presentation>
%\usetheme{default}

\usepackage{\texDirectory/template/style/directives}
%%%%%%%%%%%%%%%%%%%%%%%%%%%%%%%%%%%%%%%%%%%%%%%%%%%%%%%%%%%%%%%%%%%%%%%%%%%%%%
% CS240: Programming in C
% Copyright 2015 Pejman Ghorbanzade <pejman@ghorbanzade.com>
% Creative Commons Attribution-ShareAlike 4.0 International License
% https://github.com/ghorbanzade/UMB-CS114-2015F/blob/master/LICENSE
%%%%%%%%%%%%%%%%%%%%%%%%%%%%%%%%%%%%%%%%%%%%%%%%%%%%%%%%%%%%%%%%%%%%%%%%%%%%%%

\course{id}{CS240}
\course{name}{Programming in C}
\course{venue}{Mon/Wed, 5:30 PM - 6:45 PM}
\course{semester}{Spring 2016}
\course{department}{Department of Computer Science}
\course{university}{University of Massachusetts Boston}

\instructor{name}{Pejman Ghorbanzade}
\instructor{title}{}
\instructor{position}{Student Instructor}
\instructor{email}{pejman@cs.umb.edu}
\instructor{phone}{617-287-6419}
\instructor{office}{S-3-124B}
\instructor{office-hours}{Mon/Wed 16:00-17:30}
\instructor{address}{University of Massachusetts Boston, 100 Morrissey Blvd., Boston, MA}

\usepackage{\texDirectory/template/style/beamerthemePejman}
\doc{number}{16}
%\setbeamertemplate{footline}[text line]{}

\usepackage{booktabs}

\begin{document}

\prepareCover

\section{Structures}

\subsection{Definition}

\begin{slide}
	\begin{block}{Definition}

	A structure is a user-defined data type that defines a physically grouped list of variables under one name in a block of memory.

	\end{block}
\end{slide}

\begin{slide}
	\begin{block}{Syntax}

	\begin{terminal}
	struct @*\textit{datatype\_name}*@ {
	        @*\textit{type1 var1}*@;
	        @*\textit{type2 var2}*@;
	};
	\end{terminal}

	\end{block}
\end{slide}

\begin{slide}
	\begin{block}{Declaring a \texttt{struct}}

	\inputminted[
		fontsize=\scriptsize,
		firstline=10,
		linenos
	]{c}{\resDirectory/student.h}

	\end{block}
\end{slide}

\begin{slide}
	\begin{block}{Accessing \texttt{struct} Members}

	To access any member of a structure, we use the member access operator (\texttt{.}).

	\begin{terminal}
	struct student std;
	std.age = 10;
	\end{terminal}

	\end{block}
\end{slide}

\begin{slide}
	\begin{block}{Initializing a \texttt{struct}}

	\inputminted[
		fontsize=\scriptsize,
		firstline=10,
		linenos
	]{c}{\resDirectory/student1.c}

	\end{block}
\end{slide}

\begin{slide}
	\begin{block}{Accessing \texttt{struct} Members}

	Given a pointer to a structure, any member of that structure can be accessed using the member access operator (\texttt{->}).

	\begin{terminal}
	struct student std;
	ptr = &std;
	ptr->age = 10;
	\end{terminal}

	\end{block}
\end{slide}

\begin{slide}
	\begin{block}{Initializing a \texttt{struct}}

	\inputminted[
		fontsize=\scriptsize,
		firstline=10,
		lastline=20,
		linenos
	]{c}{\resDirectory/student2.c}

	\end{block}
\end{slide}

\begin{slide}
	\begin{block}{Initializing a \texttt{struct}}

	\inputminted[
		fontsize=\scriptsize,
		firstline=22,
		linenos
	]{c}{\resDirectory/student2.c}

	\end{block}
\end{slide}

\subsection{Example}

\begin{slide}
	\begin{figure}
	\includegraphics[width=0.7\textwidth]{\texDirectory/img/daltons.png}
	\end{figure}
\end{slide}

\begin{slide}
	\begin{block}{Objective}

	Write a program \texttt{daltons.h} that takes the height of the Dalton brothers, Joe, William, Jack and Averell, and outputs their names in ascending order of their heights.

	\end{block}
\end{slide}

\begin{slide}
	\begin{block}{Solution}

	\inputminted[
		fontsize=\scriptsize,
		firstline=10,
		linenos
	]{c}{\resDirectory/daltons.h}

	\end{block}
\end{slide}

\begin{slide}
	\begin{block}{Solution}

	\inputminted[
		fontsize=\scriptsize,
		firstline=16,
		lastline=33,
		linenos
	]{c}{\resDirectory/daltons.c}

	\end{block}
\end{slide}

\begin{slide}
	\begin{block}{Solution}

	\inputminted[
		fontsize=\scriptsize,
		firstline=38,
		lastline=55,
		linenos
	]{c}{\resDirectory/daltons.c}

	\end{block}
\end{slide}

\begin{slide}
	\begin{block}{Solution}

	\inputminted[
		fontsize=\scriptsize,
		firstline=60,
		lastline=71,
		linenos
	]{c}{\resDirectory/daltons.c}

	\end{block}
\end{slide}

\begin{slide}
	\begin{block}{Solution}

	\inputminted[
		fontsize=\scriptsize,
		firstline=76,
		lastline=81,
		linenos
	]{c}{\resDirectory/daltons.c}

	\end{block}
\end{slide}

\begin{slide}
	\begin{block}{Solution}

	\inputminted[
		fontsize=\scriptsize,
		firstline=86,
		lastline=93,
		linenos
	]{c}{\resDirectory/daltons.c}

	\end{block}
\end{slide}

\begin{slide}
	\begin{block}{Solution}

	\inputminted[
		fontsize=\scriptsize,
		firstline=98,
		lastline=104,
		linenos
	]{c}{\resDirectory/daltons.c}

	\end{block}
\end{slide}

\begin{slide}
	\begin{columns}
	\column{0.5\textwidth}
	\begin{quotation} \scriptsize \normalfont

A programmer was walking out of door for work when his wife said: ``while you are out, buy some milk.''

He never came home.

	\end{quotation}
	\column{0.5\textwidth}
	\begin{figure}
	\includegraphics[width=\textwidth]{\texDirectory/img/obama.jpg}
	\end{figure}
	\end{columns}
\end{slide}

\end{document}
