%%%%%%%%%%%%%%%%%%%%%%%%%%%%%%%%%%%%%%%%%%%%%%%%%%%%%%%%%%%%%%%%%%%%%%
% UMB-CS240-2016S: Programming in C
% Copyright 2016 Pejman Ghorbanzade <pejman@ghorbanzade.com>
% Creative Commons Attribution-ShareAlike 4.0 International License
% More info: https://github.com/ghorbanzade/UMB-CS240-2016S
%%%%%%%%%%%%%%%%%%%%%%%%%%%%%%%%%%%%%%%%%%%%%%%%%%%%%%%%%%%%%%%%%%%%%%

\section*{Question 2}

Edgar Linton, a CS240 student, was asked to write a program that promps user for an integer and prints it on the screen.
A sample run of the program was expected to be as given below.

\begin{terminal}
$ gcc prompt-number.c -o prompt-number
$ ./prompt-number
enter number: 125
number: 125
\end{terminal}

To solve this problem, he wrote the following program.

\lstset{language=c,tabsize=4}
\begin{lstlisting}
// Edgar Linton
// <edgar.linton001@umb.edu>

#include <stdio.h>
#define MAX_LENGTH 32

int main(void)
{
	int num = getNum();
	printf("%d\n", num);
}

int getNum(void)
{
	int res;
	char ch;
	int i = 0;
	char array[MAX_LENGTH];
	while ((ch = getchar()) != '\n')
		array[i] = ch;
	i = 0;
	while (array[i] != '\0')
		res = res * 10 + array[i++] - '0';
	return res;
}
\end{lstlisting}

His program produces the following output.

\begin{terminal}
$ gcc prompt-number.c -o prompt-number
$ ./prompt-number
enter number: 125
number: 971397989
\end{terminal}

Clearly, something is wrong with Edgar's code.
This is why he is asking for your help to debug his program and fix all its bugs.
Rewrite the program \texttt{prompt-number.c} such that its output matches the expected output provided previously.
