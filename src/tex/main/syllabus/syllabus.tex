%%%%%%%%%%%%%%%%%%%%%%%%%%%%%%%%%%%%%%%%%%%%%%%%%%%%%%%%%%%%%%%%%%%%%%%%%%%%%%
% CS114: Introduction to Programming in Java
% Copyright 2015 Pejman Ghorbanzade <mail@ghorbanzade.com>
% Creative Commons Attribution-ShareAlike 4.0 International License
% https://github.com/ghorbanzade/UMB-CS114-2015F/blob/master/LICENSE
%%%%%%%%%%%%%%%%%%%%%%%%%%%%%%%%%%%%%%%%%%%%%%%%%%%%%%%%%%%%%%%%%%%%%%%%%%%%%%

\def \topDirectory {.}
\def \texDirectory {\topDirectory/src/main/tex}

\documentclass[12pt,letterpaper,twoside]{article}
\usepackage{\texDirectory/template/style/directives}
\usepackage{\texDirectory/template/style/assignment}
%%%%%%%%%%%%%%%%%%%%%%%%%%%%%%%%%%%%%%%%%%%%%%%%%%%%%%%%%%%%%%%%%%%%%%%%%%%%%%
% CS240: Programming in C
% Copyright 2015 Pejman Ghorbanzade <pejman@ghorbanzade.com>
% Creative Commons Attribution-ShareAlike 4.0 International License
% https://github.com/ghorbanzade/UMB-CS114-2015F/blob/master/LICENSE
%%%%%%%%%%%%%%%%%%%%%%%%%%%%%%%%%%%%%%%%%%%%%%%%%%%%%%%%%%%%%%%%%%%%%%%%%%%%%%

\course{id}{CS240}
\course{name}{Programming in C}
\course{venue}{Mon/Wed, 5:30 PM - 6:45 PM}
\course{semester}{Spring 2016}
\course{department}{Department of Computer Science}
\course{university}{University of Massachusetts Boston}

\instructor{name}{Pejman Ghorbanzade}
\instructor{title}{}
\instructor{position}{Student Instructor}
\instructor{email}{pejman@cs.umb.edu}
\instructor{phone}{617-287-6419}
\instructor{office}{S-3-124B}
\instructor{office-hours}{Mon/Wed 16:00-17:30}
\instructor{address}{University of Massachusetts Boston, 100 Morrissey Blvd., Boston, MA}

\setlist{leftmargin=10pt, itemsep=0pt, parsep=0pt}

\begin{document}

\doc{title}{Course Syllabus}
\doc{points}{0}

\prepare{header}

\subsection*{Objectives}
This course is designed to introduce C programming language as both a general purpose and machine-level language.
It will also include introductory topics on Unix operating system as well as build automation and debugging tools for software written in C language.
It is assumed that students are already familiar with at least one high-level programming language.

\subsection*{Topics}
\begin{itemize}
\item[] Data Types, Operators and Expressions
\item[] Program Control Flow
\item[] Functions and Program Structure
\item[] Pointers and Arrays
\item[] Structures, Enums, Unions
\item[] Dynamic Memory Allocation
\item[] Dynamic Data Structures
\item[] File Handling
\end{itemize}

\subsection*{Prerequisites}
Students taking this course should have passed CS110 (Introduction to Computing) or CS115 (Introduction to Java - Part 2) and should be taking CS210 (Intermediate Computing with Data Structures) at the same time.

\subsection*{Venue}
\begin{itemize}
\item[] Classroom: M-01-208
\item[] Weekdays: Monday, Wednesday
\item[] Time: 17:30 - 18:45
\end{itemize}

\subsection*{Recommended Textbook}
Brian Kernighan and Dennis Ritchie, The C Programming Language, 2nd Edition, Prentice Hall

\subsection*{Instructor}
\begin{itemize}
\item[] Pejman Ghorbanzade
\item[] Mail Address: \texttt{mail@ghorbanzade.com}
\item[] Office: S-3-124
\item[] Office Hours: Mondays, Wednesdays 16:00 to 17:30
\end{itemize}

\subsection*{Teaching Assistants}
\begin{itemize}
\item[] Jia Shaohua
\item[] Mail Address: \texttt{shaohia@cs.umb.edu}
\item[] Office: S-3-135
\end{itemize}

\subsection*{Course Website}
Students are expected to check the course website on a regular basis.

All information such as important announcements, course schedule, useful links as well as assignments and their solution will be available at the course website.
In the unlikely case that the course website is not accessible, \textbf{some} materials can be found at the course reposiory.

\begin{itemize}
\item[] Course Website: \texttt{\footnotesize http://ghorbanzade.com/teaching/CS240-2016S}
\item[] Course Repository: \texttt{\footnotesize https://github.com/ghorbanzade/UMB-CS240-2016S}
\end{itemize}

\subsection*{Grading}
\begin{itemize}
\item[] Homeworks: 40 points
\item[] Midterm Exam: 20 points
\item[] Final Project: 20 points
\item[] Final Exam: 20 points
\end{itemize}

\subsection*{Important Dates}
\begin{itemize}
\item[] First Lecture: January 25, 2016
\item[] Midterm Exam: March 9, 2016
\item[] Last Lecture: May 11, 2016
\item[] Final Exam: May 16, 2016
\end{itemize}

\subsection*{Supplemental Instruction Sessions}
As part of the College of Science and Mathematics Freshman Success Program, Supplemental Instruction (SI) is available to all CS240 students free of charge.
During the SI sessions, the SI leader will review the material we discuss in class and will also answer any questions you may have regarding concepts or assignments.
You may attend as many or as few sessions as you want or feel that you need. Attending these sessions is encouraged.

\subsubsection*{Supplemental Instruction Leader}
\begin{itemize}
\item[] Alexander Burke
\item[] Mail Address: \texttt{Alexander.Burke001@umb@edu}
\end{itemize}

\subsubsection*{Supplemental Instruction Sessions}
\begin{itemize}
\item[] Mondays, 10:45 AM to 12:45 PM, W-1-0048
\item[] Fridays, 2:00 PM to 4:00 PM, W-1-0048
\end{itemize}

\subsection*{Attendance Policy}
Attendance will be taken for all classes but will have no effect in final evaluation.
Students are expected to participate actively in class by asking and answering questions.
Students are responsible for material covered in any class that they do not attend.

\subsection*{Late Submission Policy}
Homeworks may be submitted late by up to two days; the penalty for late submission increasing linearly from 20\% to 100\% of the homework score.

\subsection*{Makeup Policy}
Unless a good reason and its supporting evidence are given - e.g. due to illness or emergency events - no makeup is acceptable for students missing a homework assignment or an exam.

\subsection*{Accomodations}
Section 504 of the Americans with Disabilities Act of 1990 offers guidelines for curriculum modifications and adaptations for students with documented disabilities. If applicable, students may obtain adaptation recommendations from the \href{http://www.umb.edu/academics/vpass/disability}{Ross Center for Disability Services}, M-1-401, (617-287-7430). The student must present these recommendations and discuss them with each professor within a reasonable period, preferably by the end of Drop/Add period.

\subsection*{Student Conduct}
Students are required to adhere to the University Policy on Academic Standards and Cheating, to the University Statement on Plagiarism and the Documentation of Written Work, and to the \href{http://www.umb.edu/life_on_campus/policies/community/code}{Code of Student Conduct} as delineated in the catalog of Undergraduate Programs, pp. 44-45, and 48-52.

\end{document}
