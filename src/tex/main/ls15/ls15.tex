%%%%%%%%%%%%%%%%%%%%%%%%%%%%%%%%%%%%%%%%%%%%%%%%%%%%%%%%%%%%%%%%%%%%%%%%%%%%%%
% CS240: Programming in C
% Copyright 2016 Pejman Ghorbanzade <pejman@ghorbanzade.com>
% Creative Commons Attribution-ShareAlike 4.0 International License
% https://github.com/ghorbanzade/UMB-CS240-2016S/blob/master/LICENSE
%%%%%%%%%%%%%%%%%%%%%%%%%%%%%%%%%%%%%%%%%%%%%%%%%%%%%%%%%%%%%%%%%%%%%%%%%%%%%%

\def \topDirectory {.}
\def \resDirectory {\topDirectory/src/c/main/ls15}
\def \texDirectory {\topDirectory/src/tex}
\def \styDirectory {\texDirectory/sty}
\def \cfgDirectory {\texDirectory/cfg}
\def \imgDirectory {\texDirectory/img}

\documentclass[compress]{beamer}
%\mode<presentation>
%\usetheme{default}

\usepackage{\styDirectory/directives}
%%%%%%%%%%%%%%%%%%%%%%%%%%%%%%%%%%%%%%%%%%%%%%%%%%%%%%%%%%%%%%%%%%%%%%%%%%%%%%
% CS240: Programming in C
% Copyright 2015 Pejman Ghorbanzade <pejman@ghorbanzade.com>
% Creative Commons Attribution-ShareAlike 4.0 International License
% https://github.com/ghorbanzade/UMB-CS114-2015F/blob/master/LICENSE
%%%%%%%%%%%%%%%%%%%%%%%%%%%%%%%%%%%%%%%%%%%%%%%%%%%%%%%%%%%%%%%%%%%%%%%%%%%%%%

\course{id}{CS240}
\course{name}{Programming in C}
\course{venue}{Mon/Wed, 5:30 PM - 6:45 PM}
\course{semester}{Spring 2016}
\course{department}{Department of Computer Science}
\course{university}{University of Massachusetts Boston}

\instructor{name}{Pejman Ghorbanzade}
\instructor{title}{}
\instructor{position}{Student Instructor}
\instructor{email}{pejman@cs.umb.edu}
\instructor{phone}{617-287-6419}
\instructor{office}{S-3-124B}
\instructor{office-hours}{Mon/Wed 16:00-17:30}
\instructor{address}{University of Massachusetts Boston, 100 Morrissey Blvd., Boston, MA}

\usepackage{\styDirectory/beamerthemePejman}
\doc{number}{15}
%\setbeamertemplate{footline}[text line]{}

\usepackage{booktabs}

\begin{document}

\prepareCover

\section{File Handling}

\subsection{Unix File System}

\begin{slide}
	\begin{block}{Definition}

	A Unix file system is a single rooted-tree of directories which form a logical collection of files on a partition.

	\end{block}
\end{slide}

\begin{slide}
	\begin{block}{What is a File?}

	\begin{itemize}
	\item[] A file is \emph{simply} a sequence of bytes.
	\item[] The number of bytes stored in a file defines its size.
	\end{itemize}

	\end{block}
\end{slide}

\begin{slide}
	\begin{block}{What is a Directory?}

	\begin{itemize}
	\item[] A directory is a container of a set of files.
	\item[] Directories allow better organization of files.
	\item[] By origin, a directory is a file that lists the set of files that it contains.
	\end{itemize}

	\end{block}
\end{slide}

\begin{slide}
	\begin{block}{Special Files}

	In Unix, everything, including physical drives, is considered to be a file.
	Files representing external physical devices are treated as \emph{special}.

	\end{block}
\end{slide}

\begin{slide}
	\begin{block}{Lifetime}

	Files are created for permanent storage of data.
	Information stored in a file (located in a writable file system) is held until the user modifies or removes the file.

	\end{block}
\end{slide}

\subsection{File Handling in C}

\begin{slide}
	\begin{block}{Pointer to Files}

	C allows creation, modification and removal of files using a pointer to a \texttt{FILE} data type.

	\begin{terminal}
	FILE *fp;
	\end{terminal}

	\end{block}
\end{slide}

\begin{slide}
	\begin{block}{Overview}

	All input/output handlings in C are made through a set of built-in functions.

	\begin{description}
	\item[\texttt{fopen()}] create a new file or open an existing file
	\item[\texttt{fread()}] reads a number of bytes from a file
	\item[\texttt{fwrite()}] writes a number of bytes into a file
	\item[\texttt{fclose()}] closes a file to be used by other processes
	\end{description}

	\end{block}
\end{slide}

\begin{slide}
	\begin{block}{Opening a file}

	A file must be opened before reading from or writing into it.

	\begin{terminal}
	FILE *fopen(const char *filename, const char *mode);
	\end{terminal}

	\end{block}
\end{slide}

\begin{slide}
	\begin{block}{Access Modes}

	You must specify the mode in which you want to access a file.

	\begin{table}
	\begin{tabular}{cl}
	\toprule
	Mode & Description\\
	\midrule
	r & open an existing file for reading\\
	w & create or open file for writing from the beginning\\
	a & create or open file for writing from the end\\
	\bottomrule
	\end{tabular}
	\caption{File Access Modes}
	\end{table}

	\end{block}
\end{slide}

\begin{slide}
	\begin{block}{Writing to File}

	\inputminted[
		fontsize=\scriptsize,
		firstline=10,
		linenos
	]{c}{\resDirectory/fileio-w.c}

	\end{block}
\end{slide}

\begin{slide}
	\begin{block}{\texttt{write} Access Mode}

	Access mode \alert{\texttt{w}} creates the file if it does not exist.

	\begin{terminal}
	$ ls
	file-io.c
	$ gcc file-io.c
	$ ./a.out
	$ cat sample.txt
	content of the file
	\end{terminal}

	\end{block}
\end{slide}

\begin{slide}
	\begin{block}{\texttt{write} Access Mode}

	Access mode \alert{\texttt{r+}} returns a pointer to the first byte of the file.

	\begin{terminal}
	$ ls
	file-io.c sample.txt
	$ cat sample.txt
	previous content of file
	$ gcc file-io.c
	$ ./a.out
	$ ls
	a.out file-io.c sample.txt
	$ cat sample.txt
	content of the file
	previous content of file
	\end{terminal}

	\end{block}
\end{slide}

\begin{slide}
	\begin{block}{\texttt{append} Access Mode}

	Access mode \alert{\texttt{a}} returns a pointer to the last byte of the file.

	\begin{terminal}
	$ ls
	file-io.c sample.txt
	$ cat sample.txt
	previous content of file
	$ gcc file-io.c
	$ ./a.out
	$ ls
	a.out file-io.c sample.txt
	$ cat sample.txt
	previous content of file
	content of the file
	\end{terminal}

	\end{block}
\end{slide}

\begin{slide}
	\begin{block}{Reading from File}

	\inputminted[
		fontsize=\scriptsize,
		firstline=10,
		linenos
	]{c}{\resDirectory/fileio-r1.c}

	\end{block}
\end{slide}

\begin{slide}
	\begin{block}{\texttt{read} Access Mode}

	Access mode \alert{\texttt{r}} returns a pointer to the first byte of the file.

	\begin{terminal}
	$ ls
	file-io.c sample.txt
	$ cat sample.txt
	content of file
	$ gcc file-io.c
	$ ./a.out
	content of file
	\end{terminal}

	\end{block}
\end{slide}

\begin{slide}
	\begin{block}{\texttt{fgets()} Function}

	\inputminted[
		fontsize=\scriptsize,
		firstline=10,
		linenos
	]{c}{\resDirectory/fileio-r2.c}

	\end{block}
\end{slide}

\begin{slide}
	\begin{block}{\texttt{fread()} Function}

	\inputminted[
		fontsize=\scriptsize,
		firstline=10,
		linenos
	]{c}{\resDirectory/fileio-r3.c}

	\end{block}
\end{slide}

\begin{slide}
	\begin{block}{\texttt{fclose() Function}}

	\texttt{fclose()} is used to close an already opened file and release its lock, if any.

	\begin{terminal}
	int fclose(FILE *fp)
	\end{terminal}

	Pointer passed to \texttt{fclose()} must be the pointer returned by a successful call to \texttt{fopen()}.

	\end{block}
\end{slide}

\begin{slide}
	\begin{figure}
	\includegraphics[width=0.7\textwidth]{\imgDirectory/fread.jpg}
	\end{figure}
\end{slide}

\end{document}
