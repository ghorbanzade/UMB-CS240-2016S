%%%%%%%%%%%%%%%%%%%%%%%%%%%%%%%%%%%%%%%%%%%%%%%%%%%%%%%%%%%%%%%%%%%%%%
% UMB-CS240-2016S: Programming in C
% Copyright 2016 Pejman Ghorbanzade <pejman@ghorbanzade.com>
% Creative Commons Attribution-ShareAlike 4.0 International License
% More info: https://github.com/ghorbanzade/UMB-CS240-2016S
%%%%%%%%%%%%%%%%%%%%%%%%%%%%%%%%%%%%%%%%%%%%%%%%%%%%%%%%%%%%%%%%%%%%%%

\section*{Question 2}

Write a program \texttt{slow-prime.c} that takes a positive integer number as a command line argument and prints the smallest prime number bigger than the given number.
The \texttt{main()} function of program is given below and you may not modify it.
In addition, you may not include any header file other than \texttt{slow-prime.h} in \texttt{slow-prime.c}.

\lstset{language=c,tabsize=4}
\lstinputlisting[firstline=16,lastline=34]{\resDirectory/slow-prime.c}

Following is the expected output of a sample run of your program.

\begin{terminal}
$ ls
slow-prime.c slow-prime.h
$ gcc -c -o slow-prime.o slow-prime.c
$ gcc -o slow-prime slow-prime.o -Werror -Wall -std=gnu99 -I.
$ ./slow-prime
error: missing command line argument
$ ./slow-prime 15sa
error: 15sa not a number
$ ./slow-prime 19
23
\end{terminal}
