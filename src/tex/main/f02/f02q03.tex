%%%%%%%%%%%%%%%%%%%%%%%%%%%%%%%%%%%%%%%%%%%%%%%%%%%%%%%%%%%%%%%%%%%%%%
% UMB-CS240-2016S: Programming in C
% Copyright 2016 Pejman Ghorbanzade <pejman@ghorbanzade.com>
% Creative Commons Attribution-ShareAlike 4.0 International License
% More info: https://github.com/ghorbanzade/UMB-CS240-2016S
%%%%%%%%%%%%%%%%%%%%%%%%%%%%%%%%%%%%%%%%%%%%%%%%%%%%%%%%%%%%%%%%%%%%%%

\section*{Question 3}

The C programming language, as of C99, supports complex number math with the three built-in types \texttt{double \_Complex}, \texttt{float \_Complex}, and \texttt{long double \_Complex}.
When the header \texttt{<complex.h>} is included, the three complex number types are also accessible as \texttt{double complex}, \texttt{float complex}, \texttt{long double complex}.
Standard arithmetic operators \texttt{+}, \texttt{-}, \texttt{*}, \texttt{/} can be used with real, complex, and imaginary types in any combination.

The following program shows how complex numbers can be used in C. It loads components of complex numbers from a file and prints their sum on standard output.

\lstset{language=c,tabsize=4}
\lstinputlisting[firstline=10]{\resDirectory/complex-numbers.h}
\lstinputlisting[firstline=10]{\resDirectory/complex-numbers.c}

The program compiles and executes as expected.
Based on the given source code, provide \textbf{brief} answers for the following questions.

\begin{enumerate}
\item
In \texttt{complex-numbers.h:10}, function \texttt{cprint} is declared but not defined.
What is the advantage of declaring a function in a header file?
Explain why removing this line causes a compilation error and how the error can be resolved without including this line.

\item
In \texttt{complex-numbers.c:13}, \texttt{fp} is checked to make sure it is not \texttt{null}.
A CS240 student argues this checking is unnecessary if we make sure the file \texttt{complex-numbers.txt} exists.
Describe whether you support this argument or not and provide brief explanations for your reasoning.

\item
In \texttt{complex-numbers.c:17}, variables \texttt{a} and \texttt{b} are passed by reference to function \texttt{scanf}.
Explain what happens if they are passed by value.

\item
In \texttt{complex-numbers.c:18}, \texttt{I} is used but it is not explicitly declared either in \texttt{complex-numbers.c} or \texttt{complex-numbers.h}.
This also applies to functions \texttt{creal} and \texttt{cimag} in \texttt{complex-numbers.c:31}.
Briefly explain why such practice has not caused compilation error.

\item
In \texttt{complex-numbers.c:21}, variable \texttt{sum} is passed to \texttt{cprint} by value. Modify this line and the function \texttt{cprint} such that \texttt{num} is passed by reference.

\end{enumerate}
