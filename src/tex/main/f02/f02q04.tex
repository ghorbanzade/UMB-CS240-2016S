%%%%%%%%%%%%%%%%%%%%%%%%%%%%%%%%%%%%%%%%%%%%%%%%%%%%%%%%%%%%%%%%%%%%%%%%%%%%%%
% CS240: Programming in C
% Copyright 2015 Pejman Ghorbanzade <pejman@ghorbanzade.com>
% Creative Commons Attribution-ShareAlike 4.0 International License
% https://github.com/ghorbanzade/UMB-CS240-2016S/blob/master/LICENSE
%%%%%%%%%%%%%%%%%%%%%%%%%%%%%%%%%%%%%%%%%%%%%%%%%%%%%%%%%%%%%%%%%%%%%%%%%%%%%%

\section*{Question 4}

Since fall of You-Know-Who, Parvati Patil has been working as payroll clerk at Ministry of Magic.
She is responsible for preparing and reporting of payroll checks.
To prevent fraud and curruption, the Ministry has prohibited any use of magic in department of Human Resources.
Parvati is therefore required to calculate all payrolls as Muggles do.
Given the number of employees working at the Ministry, she has been thinking to write a software that automates payroll calculations for all employees.

Depending on their status, employees have different pay rates in the Ministry as given in Table \ref{tablequestion5}.
As is shown, part time and interns are paid by hour whereas full time employees are paid regardless of the number of hours they have worked.

You are asked to help in design of the payroll system for the Ministry.
Each employee has a name, an ID, the number of hours he has worked in a month and a status which is either full time, part time or intern.

\begin{enumerate}
\item Define a structure \texttt{employee} to enclose required information for a given employee.
\item Write a function that takes data of an employee and returns an initialized employee structure.
\item Write a function that takes an employee structure and calculates the monthly pay for that employee.
\item Write a function that takes an array of employee structures and returns the total amount to be paid to those employees.
\end{enumerate}

\begin{table}[H] \centering
\begin{tabular}{r c}
\textbf{Status} & \textbf{Rate}\\
\hline
Full Time & 800 Galleons / Month\\
Part Time & 4 Galleons / Hour\\
Intern & 2 Galleons / Hour
\end{tabular}
\caption{Pay rates for employees of Ministry of Magic}
\label{tablequestion5}
\end{table}
