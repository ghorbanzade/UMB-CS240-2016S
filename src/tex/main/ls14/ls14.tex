%%%%%%%%%%%%%%%%%%%%%%%%%%%%%%%%%%%%%%%%%%%%%%%%%%%%%%%%%%%%%%%%%%%%%%%%%%%%%%
% CS240: Programming in C
% Copyright 2016 Pejman Ghorbanzade <mail@ghorbanzade.com>
% Creative Commons Attribution-ShareAlike 4.0 International License
% https://github.com/ghorbanzade/UMB-CS240-2016S/blob/master/LICENSE
%%%%%%%%%%%%%%%%%%%%%%%%%%%%%%%%%%%%%%%%%%%%%%%%%%%%%%%%%%%%%%%%%%%%%%%%%%%%%%

\def \topDirectory {../..}
\def \texDirectory {\topDirectory/src/main/tex}
\def \resDirectory {\topDirectory/src/main/c/ls14}

\documentclass[compress]{beamer}
%\mode<presentation>
%\usetheme{default}

\usepackage{\texDirectory/template/style/directives}
%%%%%%%%%%%%%%%%%%%%%%%%%%%%%%%%%%%%%%%%%%%%%%%%%%%%%%%%%%%%%%%%%%%%%%%%%%%%%%
% CS240: Programming in C
% Copyright 2015 Pejman Ghorbanzade <pejman@ghorbanzade.com>
% Creative Commons Attribution-ShareAlike 4.0 International License
% https://github.com/ghorbanzade/UMB-CS114-2015F/blob/master/LICENSE
%%%%%%%%%%%%%%%%%%%%%%%%%%%%%%%%%%%%%%%%%%%%%%%%%%%%%%%%%%%%%%%%%%%%%%%%%%%%%%

\course{id}{CS240}
\course{name}{Programming in C}
\course{venue}{Mon/Wed, 5:30 PM - 6:45 PM}
\course{semester}{Spring 2016}
\course{department}{Department of Computer Science}
\course{university}{University of Massachusetts Boston}

\instructor{name}{Pejman Ghorbanzade}
\instructor{title}{}
\instructor{position}{Student Instructor}
\instructor{email}{pejman@cs.umb.edu}
\instructor{phone}{617-287-6419}
\instructor{office}{S-3-124B}
\instructor{office-hours}{Mon/Wed 16:00-17:30}
\instructor{address}{University of Massachusetts Boston, 100 Morrissey Blvd., Boston, MA}

\usepackage{\texDirectory/template/style/beamerthemePejman}
\doc{number}{14}
%\setbeamertemplate{footline}[text line]{}

\begin{document}

\prepareCover

\section{Memory Management}

\subsection{Static Memory Allocation}

\begin{slide}
	\begin{block}{Introduction}

	Static memory is the memory allocated at compile-time before the associated program is executed.


	\end{block}
\end{slide}

\begin{slide}
	\begin{block}{Lifetime}

	Static variables persist during the lifetime of the program and are accessible from within any function defined in the same file in which the variable is declared.

	\end{block}
\end{slide}

\begin{slide}
	\begin{block}{Syntax}

	Static memory can be allocated using the \texttt{static} keyword.

	\begin{terminal}
	static int a = 4;
	\end{terminal}

	\end{block}
\end{slide}

\begin{slide}
	\begin{block}{Example}

	\inputminted[
		fontsize=\scriptsize,
		firstline=10,
		linenos
	]{c}{\resDirectory/static.c}

	\end{block}
\end{slide}

\subsection{Atomatic Memory Allocation}

\begin{slide}
	\begin{block}{Introduction}

	Automatic memory is the memory allocated in the \alert{stack} frame of the function they are declared in.

	\end{block}
\end{slide}

\begin{slide}
	\begin{block}{Lifetime}

	Automatic variables have a scope \emph{local} to the function they are declared in.
	Memory allocated to an automatic variable is freed once its declaring function exits the stack (returns).

	\end{block}
\end{slide}

\begin{slide}
	\begin{block}{Example}

	\inputminted[
		fontsize=\scriptsize,
		firstline=10,
		linenos
	]{c}{\resDirectory/automatic.c}

	\end{block}
\end{slide}

\begin{slide}
	\begin{block}{Limitation}

	Size of the memory to be allocated for static or automatic variables must be constant and known to the compiler.

	\end{block}
\end{slide}

\begin{slide}
	\begin{block}{Problem}

	What if the required size for a variable is not known during the compilation time?

	\end{block}
\end{slide}

\subsection{Dynamic Memory Allocation}

\begin{slide}
	\begin{block}{Objective}

	Write a program \texttt{dynamic-array.c} that continues to prompt user for integers as long as the number is not negative.
	Your program should print all numbers previously entered, once the user enters a negative integer.

	\end{block}
\end{slide}

\begin{slide}
	\begin{block}{Introduction}

	C allows \emph{dynamic memory allocation} with which memory allocated to a variable can be explicitly manipulated.
	Memory dynamically assigned to a variable is \emph{borrowed} from \emph{the heap}.

	\end{block}
\end{slide}

\begin{slide}
	\begin{block}{Heap Memory}

	\textit{The heap} is a region in computer memory which is not tightly managed by the CPU.
	The heap memory may be manually managed by the programmer.

	\end{block}
\end{slide}

\begin{slide}
	\begin{block}{Allocating Heap Memory}

	C provides a set of built-in functions to allocate memory on the heap.
	If the allocation is successful, the address of the allocated memory can be stored in a pointer and used the same way as the memory allocated on the stack.

	\end{block}
\end{slide}

\begin{slide}
	\begin{block}{\texttt{malloc()} Function}

	Declaring an array of fixed-size 10 on the stack memory.

	\begin{terminal}
	int array[10];
	\end{terminal}

	Declaring an array with initial size 10 on the heap memory.

	\begin{terminal}
	int *array = malloc(10 * sizeof(int));
	\end{terminal}

	\end{block}
\end{slide}

\begin{slide}
	\begin{block}{Note}

	A request for allocated memory of a certain size may be rejected by the operating system, in which case, \texttt{malloc()} will return \texttt{NULL}.

	\end{block}
\end{slide}

\begin{slide}
	\begin{block}{Error Checking}

	\begin{terminal}
	int *array = malloc(10 * sizeof(int));
	if (array == NULL)
		goto ERROR;
	\end{terminal}

	\end{block}
\end{slide}


\begin{slide}
	\begin{block}{Memory Leak}

	Memory allocated on the heap will persist as long as it is not freed.
	If a program terminates without freeing a memory block reserved on the heap for it, the memory block will remain inaccessible by other processes, causing \emph{memory leak}.

	\end{block}
\end{slide}

\begin{slide}
	\begin{block}{\texttt{free()} Function}

	The function \texttt{void free(void *ptr)} deallocates the heap memory previously allocated by the program.

	\end{block}
\end{slide}

\begin{slide}
	\begin{block}{Note}

	Function \texttt{void *malloc(size\_t size)} does not initialize the allocated memory.

	\end{block}
\end{slide}

\begin{slide}
	\begin{block}{\texttt{calloc()} Function}

	Function \texttt{void *calloc(size\_t num, size\_t size)} can be used to dynamically allocate \texttt{num} elements of size \texttt{size} on the heap and initialize all the elements with zero.

	\begin{terminal}
	int *array = calloc(10, sizeof(int));
	if (array == NULL)
		goto ERROR;
	\end{terminal}

	\end{block}
\end{slide}

\begin{slide}
	\begin{block}{Note}

	Since \texttt{calloc()} initializes the allocated memory, a call to \texttt{calloc()} is slightly more costly than a call to \texttt{malloc()}.

	\end{block}
\end{slide}

\begin{slide}
	\begin{block}{Example}

	\inputminted[
		fontsize=\scriptsize,
		firstline=10,
		linenos
	]{c}{\resDirectory/malloc.c}

	\end{block}
\end{slide}

\begin{slide}
	\begin{block}{Memory Reallocation}

	A dynamically allocated memory block can be reallocated to another memory block of a different size.

	\end{block}
\end{slide}

\begin{slide}
	\begin{block}{Function \texttt{realloc()}}

	C provides the \texttt{realloc()} function for memory reallocation.

	\begin{terminal}
	void *realloc(void *ptr, size_t size);
	\end{terminal}

	\end{block}
\end{slide}

\begin{slide}
	\begin{block}{Function \texttt{realloc()}}

	\begin{terminal}
	int *ptr1 = malloc(5 * sizeof(int));
	size_t size = 10 * sizeof(int);
	int *ptr2 = realloc(ptr1, size);
	\end{terminal}

	If reallocation is successful:
	\begin{itemize}
	\item[] A new memory block of size \texttt{size} will be allocated.
	\item[] Content of memory block pointed by \texttt{ptr1} will be copied to the new memory block.
	\item[] Memory block pointed by \texttt{ptr1} is deallocated.
	\item[] Function returns address of newly allocated memory block.
	\end{itemize}

	\end{block}
\end{slide}

\begin{slide}
	\begin{block}{Function \texttt{realloc()}}

	\begin{terminal}
	int *ptr1 = malloc(5 * sizeof(int));
	size_t size = 10 * sizeof(int);
	int *ptr2 = realloc(ptr1, size);
	\end{terminal}

	If reallocation fails:
	\begin{itemize}
	\item[] Memory block pointed by \texttt{ptr1} remains allocated.
	\item[] Function returns \texttt{NULL}.
	\end{itemize}

	\end{block}
\end{slide}

\begin{slide}
	\begin{block}{Example}

	\inputminted[
		fontsize=\scriptsize,
		firstline=10,
		linenos
	]{c}{\resDirectory/realloc.c}

	\end{block}
\end{slide}

\begin{slide}
	\begin{block}{Notes}

	\begin{itemize}
	\item[] Accessing the memory allocated on the heap is slightly slower than the memory allocated on the stack.
	\item[] Exessive heap memory allocation may lead to memory fragmentation.
	\end{itemize}

	\end{block}
\end{slide}

\begin{slide}
	\begin{figure}
	\includegraphics[width=0.5\textwidth]{\texDirectory/img/malloc.jpg}
	\end{figure}
\end{slide}

\end{document}
