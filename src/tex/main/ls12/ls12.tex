%%%%%%%%%%%%%%%%%%%%%%%%%%%%%%%%%%%%%%%%%%%%%%%%%%%%%%%%%%%%%%%%%%%%%%%%%%%%%%
% CS240: Programming in C
% Copyright 2016 Pejman Ghorbanzade <pejman@ghorbanzade.com>
% Creative Commons Attribution-ShareAlike 4.0 International License
% https://github.com/ghorbanzade/UMB-CS240-2016S/blob/master/LICENSE
%%%%%%%%%%%%%%%%%%%%%%%%%%%%%%%%%%%%%%%%%%%%%%%%%%%%%%%%%%%%%%%%%%%%%%%%%%%%%%

\def \topDirectory {.}
\def \resDirectory {\topDirectory/src/c/main/ls12}
\def \texDirectory {\topDirectory/src/tex}
\def \styDirectory {\texDirectory/sty}
\def \cfgDirectory {\texDirectory/cfg}
\def \imgDirectory {\texDirectory/img}

\documentclass[compress]{beamer}
%\mode<presentation>
%\usetheme{default}

\usepackage{\styDirectory/directives}
%%%%%%%%%%%%%%%%%%%%%%%%%%%%%%%%%%%%%%%%%%%%%%%%%%%%%%%%%%%%%%%%%%%%%%%%%%%%%%
% CS240: Programming in C
% Copyright 2015 Pejman Ghorbanzade <pejman@ghorbanzade.com>
% Creative Commons Attribution-ShareAlike 4.0 International License
% https://github.com/ghorbanzade/UMB-CS114-2015F/blob/master/LICENSE
%%%%%%%%%%%%%%%%%%%%%%%%%%%%%%%%%%%%%%%%%%%%%%%%%%%%%%%%%%%%%%%%%%%%%%%%%%%%%%

\course{id}{CS240}
\course{name}{Programming in C}
\course{venue}{Mon/Wed, 5:30 PM - 6:45 PM}
\course{semester}{Spring 2016}
\course{department}{Department of Computer Science}
\course{university}{University of Massachusetts Boston}

\instructor{name}{Pejman Ghorbanzade}
\instructor{title}{}
\instructor{position}{Student Instructor}
\instructor{email}{pejman@cs.umb.edu}
\instructor{phone}{617-287-6419}
\instructor{office}{S-3-124B}
\instructor{office-hours}{Mon/Wed 16:00-17:30}
\instructor{address}{University of Massachusetts Boston, 100 Morrissey Blvd., Boston, MA}

\usepackage{\styDirectory/beamerthemePejman}
\doc{number}{12}
%\setbeamertemplate{footline}[text line]{}

\begin{document}

\prepareCover

\section{Pointers}

\subsection{Array of Pointers}

\begin{slide}
	\begin{block}{Introduction}

	\begin{itemize}
	\item[] Array is a data structure that holds a \emph{fixed} number of variables of a \emph{single} type.
	\item[] A pointer is a \alert{variable} that points to another variable.
	\item[] An array can hold a fixed number of pointers of similar type.
	\end{itemize}

	\end{block}
\end{slide}

\begin{slide}
	\begin{block}{Declaration}

	\inputminted[
		fontsize=\scriptsize,
		firstline=10,
		linenos
	]{c}{\resDirectory/array1.c}

	\end{block}
\end{slide}

\begin{slide}
	\begin{block}{Array of Strings}

	\inputminted[
		fontsize=\scriptsize,
		firstline=10,
		linenos
	]{c}{\resDirectory/array2.c}

	\end{block}
\end{slide}

\begin{slide}
	\begin{block}{Array of Strings}

	\inputminted[
		fontsize=\scriptsize,
		firstline=10,
		linenos
	]{c}{\resDirectory/array3.c}

	\end{block}
\end{slide}

\begin{slide}
	\begin{block}{Pointers vs. Multi-dimensional Arrays}

	\inputminted[
		fontsize=\scriptsize,
		firstline=10,
		linenos
	]{c}{\resDirectory/array4.c}

	\end{block}
\end{slide}

\begin{slide}
	\begin{block}{Command-line Arguments}

	\inputminted[
		fontsize=\scriptsize,
		firstline=10,
		linenos
	]{c}{\resDirectory/cmdargs.c}

	\end{block}
\end{slide}

\subsection{Pointers to Pointers}

\begin{slide}
	\begin{figure}
	\includegraphics[width=0.6\textwidth]{\imgDirectory/pointers2.jpg}
	\end{figure}
\end{slide}

\begin{slide}
	\begin{block}{Introduction}

	A pointers may point to another pointer.

	\inputminted[
		fontsize=\scriptsize,
		firstline=10,
		linenos
	]{c}{\resDirectory/ptrptr1.c}

	\end{block}
\end{slide}

\subsection{Pointer to Functions}

\begin{slide}
	\begin{block}{Introduction}

	A function in C is a pointer to a memory location where its implementation is stored.

	\end{block}
\end{slide}

\begin{slide}
	\begin{block}{How it works!}

	\inputminted[
		fontsize=\scriptsize,
		firstline=10,
		lastline=20,
		linenos
	]{c}{\resDirectory/ptrfunc1.c}

	\end{block}
\end{slide}

\begin{slide}
	\begin{block}{How it works!}

	\inputminted[
		fontsize=\scriptsize,
		firstline=22,
		linenos
	]{c}{\resDirectory/ptrfunc1.c}

	\end{block}
\end{slide}

\begin{slide}
	\begin{block}{Motivation}

	\begin{itemize}
	\item[] Callback Mechanism
	\item[] Event Listeners
	\item[] Dependency Inversion
	\end{itemize}

	\end{block}
\end{slide}

\begin{slide}
	\begin{block}{Function Pointers}

	\inputminted[
		fontsize=\scriptsize,
		firstline=10,
		lastline=20,
		linenos
	]{c}{\resDirectory/ptrfunc2.c}

	\end{block}
\end{slide}

\begin{slide}
	\begin{block}{Function Pointers}

	\inputminted[
		fontsize=\scriptsize,
		firstline=22,
		linenos
	]{c}{\resDirectory/ptrfunc2.c}

	\end{block}
\end{slide}

\begin{slide}
	\begin{figure}
	\includegraphics[width=0.6\textwidth]{\imgDirectory/pulp.jpg}
	\end{figure}
\end{slide}

\end{document}
