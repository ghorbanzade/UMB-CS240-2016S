%%%%%%%%%%%%%%%%%%%%%%%%%%%%%%%%%%%%%%%%%%%%%%%%%%%%%%%%%%%%%%%%%%%%%%%%%%%%%%
% CS240: Programming in C
% Copyright 2015 Pejman Ghorbanzade <mail@ghorbanzade.com>
% Creative Commons Attribution-ShareAlike 4.0 International License
% https://github.com/ghorbanzade/UMB-CS240-2016S/blob/master/LICENSE
%%%%%%%%%%%%%%%%%%%%%%%%%%%%%%%%%%%%%%%%%%%%%%%%%%%%%%%%%%%%%%%%%%%%%%%%%%%%%%

\def \topDirectory {.}
\def \texDirectory {\topDirectory/src/main/tex}
\def \resDirectory {\topDirectory/src/main/c/hw02}

\documentclass[12pt,letterpaper,twoside]{article}
\usepackage{\texDirectory/template/style/directives}
\usepackage{\texDirectory/template/style/assignment}
%%%%%%%%%%%%%%%%%%%%%%%%%%%%%%%%%%%%%%%%%%%%%%%%%%%%%%%%%%%%%%%%%%%%%%%%%%%%%%
% CS114: Introduction to Programming in Java
% Copyright 2015 Pejman Ghorbanzade <mail@ghorbanzade.com>
% Creative Commons Attribution-ShareAlike 4.0 International License
% https://github.com/ghorbanzade/UMB-CS114-2015F/blob/master/LICENSE
%%%%%%%%%%%%%%%%%%%%%%%%%%%%%%%%%%%%%%%%%%%%%%%%%%%%%%%%%%%%%%%%%%%%%%%%%%%%%%

\course{id}{CS240}
\course{name}{Programming in C}
\course{venue}{Mon/Wed, 5:30 PM - 6:45 PM}
\course{semester}{Spring 2016}
\course{department}{Department of Computer Science}
\course{university}{University of Massachusetts Boston}

\instructor{name}{Pejman Ghorbanzade}
\instructor{title}{}
\instructor{position}{Student Instructor}
\instructor{email}{pejman@cs.umb.edu}
\instructor{phone}{617-287-6419}
\instructor{office}{S-3-124B}
\instructor{office-hours}{Mon/Wed 16:00-17:30}
\instructor{address}{University of Massachusetts Boston, 100 Morrissey Blvd., Boston, MA}


\usepackage{amsmath}

\begin{document}

\doc{title}{Solution to Assignment 2}
\doc{date-pub}{Feb 8, 2016 at 5:30 PM}
\doc{date-due}{Feb 22, 2016 at 5:30 PM}
\doc{points}{8}

\prepare{header}

\section*{Question 1}

Write a program \texttt{uppercase.c} that takes an arbitrary number of command line arguments and prints them all in uppercase.
Following examples illustrate how your program is expected to function.

\begin{terminal}
$ gcc uppercase.c -o uppercase
$ ./uppercase the taste of life was sweet
THE TASTE OF LIFE WAS SWEET
$ ./uppercase as rain upon my tongue
AS RAIN UPON MY TONGUE
$ ./uppercase
error: missing command line arguments
\end{terminal}

\subsection*{Solution}

\lstset{language=c,tabsize=4}
\lstinputlisting[firstline=10]{\resDirectory/uppercase.c}

\section*{Question 2}

Write a program \texttt{hexa.c} that takes a string in form of a command line argument and prints values of its characters in hexadecimal system.

Following is the expected output of a sample run of your program.

\begin{terminal}
$ gcc hexa.c -o hexa
$ ./hexa Boston
42 6f 73 74 6f 6e
$ ./hexa Quincy Boston
51 75 69 6e 63 79
$ ./hexa
error: missing command line argument
\end{terminal}

\subsection*{Solution}

\lstset{language=c,tabsize=4}
\lstinputlisting[firstline=10]{\resDirectory/hexa.c}

\section*{Question 3}

Write a program \texttt{sequence.c} that without using the multiply operator \texttt{*}, prints numbers $n = 2^{i}-1$ where $0 \leq i \leq 64$ separated by a single space.

\subsection*{Solution}

\lstset{language=c,tabsize=4}
\lstinputlisting[firstline=10]{\resDirectory/sequence.c}

\end{document}
